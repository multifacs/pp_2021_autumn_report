\documentclass{report}

\usepackage[T2A]{fontenc}
\usepackage[utf8]{luainputenc}
\usepackage[english, russian]{babel}
\usepackage[pdftex]{hyperref}
\usepackage[14pt]{extsizes}
\usepackage{listings}
\usepackage{color}
\usepackage{geometry}
\usepackage{enumitem}
\usepackage{multirow}
\usepackage{indentfirst}

\geometry{a4paper,top=2cm,bottom=3cm,left=2cm,right=1.5cm}
\setlength{\parskip}{0.5cm}
\setlist{nolistsep, itemsep=0.3cm, parsep=0pt}

\lstset{
    language=C++,
	basicstyle=\footnotesize,
	keywordstyle=\color{blue}\ttfamily,
	stringstyle=\color{red}\ttfamily,
	commentstyle=\color{green}\ttfamily,
	morecomment=[l][\color{magenta}]{\#}, 
	tabsize=4,
	breaklines=true,
	breakatwhitespace=true,
	title=\lstname,       
}

\begin{document}

\begin{titlepage}

    \begin{center}
    Министерство образования и науки Российской Федерации\\
    Федеральное государственное автономное образовательное учреждение  \\
    высшего образования\\
    \textbf{<<Национальный исследовательский\\Нижегородский государственный университет им. Н.И. Лобачевского>>}\\
    \textbf{(ННГУ)}\\[0.5cm]%[4.5cm]
    \textbf{Институт информационных технологий математики и механики}\\[0.5cm]
    \textbf{\large Отчет по лабораторной работе} \\[0.6cm] % название работы, затем отступ 0,6см
    \textbf{Тема:}\\
    \textbf{\large «Вычисление многомерных интегралов с использованием многошаговой схемы (метод трапеций).»}\\[2.5cm]

    \vspace{4em}

    \begin{flushright}
    \begin{minipage}{0.55\textwidth}
    \begin{flushleft}
        \textbf{Выполнил:} \\
        студент группы 381906-2\\
        Саблин А.В.\\
        \textbf{Проверил:}\\ доцент кафедры МОСТ, \\ кандидат технических наук \\ Сысоев А.В.\\
    \end{flushleft}
    \end{minipage}
    \end{flushright}
    
    \vfill
 
    Нижний Новгород \\ 2021

\end{center}
\end{titlepage}

\setcounter{page}{2}

% Содержание
\tableofcontents
\newpage

% Введение
\section*{Введение}
\addcontentsline{toc}{section}{Введение}
Задача вычисления определенного интеграла I для некоторой заданной на отрезке [a, b] функции f(x) является классической задачей математического анализа.
\par Зачастую приходится использовать приближенные формулы для вычисления определенного интеграла на основе значений подынтегральной функции f(x), так как нахождение значения интеграла с помощью формулы Ньютона - Лейбница $\int_a^b f(x)dx$ = F(b) - F(a) чревато некоторыми вычислительными сложностями. Такие специальные приближенные формулы называют квадратурными формулами или формулами численного интегрирования.
\par Большинство методов численного интегрирования, объединены общим принципом: область интегрирования разбивается по каждой из осей на равное количество частей. В каждой из полученных маленьких областей интеграл приближается простой функцией (например, линейной), значение которой вычисляется явно (путем вычисления подынтегрального выражения в одной или нескольких точках). Ввиду линейности оператора интегрирования по областям полученные значения суммируются и представляют собой результат интегрирования.
\par К данному типу относятся одномерные метод прямоугольников, метод трапеций, метод парабол (метод Симпсона). Перечисленные методы обобщается и для многомерного случая.

\newpage

% Постановка задачи
\section*{Постановка задачи}
\addcontentsline{toc}{section}{Постановка задачи}
В рамках лабораторной работы необходимо реализовать последовательную и параллельную версии реализации вычисления приближенного значения многомерных (в случае данной реализации - двумерных) интегралов при помощи метода трапеций. Для реализации параллельной версии метода нужно использовать библиотеку межпроцессорного взаимодействия MPI.
\par А также:
\begin{itemize}
    \item Протестировать алгоритмы с помощью тестов, реализованных на Google C++ Testing Framework. 
    \item Провести эксперименты для оценки эффективности параллелизации и по полученным результатам сделать выводы.
\end{itemize}

\newpage

% Описание алгоритма
\section*{Описание алгоритма}
\addcontentsline{toc}{section}{Описание алгоритма}
Так как концепция она и та же, проясним смысл алгоритма на примере одномерного определенного интеграла. Итак, предположим, что наша цель состоит в нахождении приближенного численного значения определенного интеграла $\int_a^b f(x)dx$, при условии, что функция $y = f(x)$ непрерывна на отрезке $[a;b]$. Для этого разделим отрезок $[a;b]$ на несколько равных отрезков в количестве n. Длина отрезка - $h$, вычисляется по формуле $h=\frac{b-a}{n}$. Тогда имеет место разбиение: $a=x0<x_1<x_2<...<x_{n-1}<x_n=b$.
\par Далее нам остается лишь определить узлы из равенства $x_i=a+i * h$, где $i=0, 1,..., n$.
\par Рассмотрим множество элементарных отрезков, а именно множество узлов $[x_{i-1}; x_i]$, $i=1, 2,.., n$ и вычислим приближенное значение исходного определенного интеграла с помощью этой формулы: $h*\sum\limits_{i=1}^n\frac{(f(x_{i-1})+f(x_i))}{2}$.
\newpage

\section*{Описание схемы распараллеливания}
\addcontentsline{toc}{section}{Описание схемы распараллеливания}
В лабораторной работе осуществляется распараллеливание многомерного интеграла на заданное число процессов. 
\par Каждый процесс получает пределы интегрирования функции, количество отрезков разбиения и подынтегральную функцию. После процессы вычисляют величины $h_x$, $h_y$ и принимаются к определению множества точек. Представим поле наших вычислений в виде сетки (пересечение множества $a_x=x_0<x_1<x_2<...<x_{n-1}<x_{n}=b_x$ и множества $a_y=x_0<y_1<y_2<...<y_{n-1}<y_{n}=b_y$), тогда нам нужно определить каждый узел этой сетки (вычислить функцию в точке), умножить его на $h$ и поделить на $1/2^j$, где $j$ зависит от количества примыкающих к узлу клеток (1-клетка: $j=2$, 2-клетки: $j=1$ и 4-клетки $j=0$). Все полученные результаты нужно сложить и получится приближенное значение интеграла. По сетке мы будем идти по $x$-ам, которые разделяются между процессами посредством начальной точки счёта, которая вычисляется для каждого процесса по формуле: $x = a_x + h_x * (ProcRank + 1)$, и сдвига равного количеству процессов $ProcRank$.
\newpage

% Описание программной реализации
\section*{Описание программной реализации}
\addcontentsline{toc}{section}{Описание программной реализации}
	Программа состоит из заголовочного файла trapezoid\_method.h, в котором объявлены функции:
	\begin{lstlisting}
	double trapezoidMethodSequential(double (*f)(double, double), double a_x, double b_x, double a_y, double b_y, const int n);\end{lstlisting}\par- вычисление многомерного интеграла методом трапеций, последовательная версия.
    \begin{lstlisting}
	double trapezoidMethodParallel(double (*f)(double, double), double a_x, double b_x, double a_y, double b_y, const int n);\end{lstlisting}\par- вычисление многомерного интеграла методом трапеций, параллельная версия 
	\par Где double (*f)(double, double) - подынтегральная функция;  double a\_x, double b\_x, double a\_y, double b\_y - пределы интегрирования; const int n - количество отрезков разбиения.
	\par Также программа состоит из файла trapezoid\_method.cpp в котором содержится реализация функций и файла main.cpp в котором содержаться тесты для проверки корректности программы.
\newpage

% Подтверждение корректности
\section*{Подтверждение корректности}
\addcontentsline{toc}{section}{Подтверждение корректности}
Корректность работы программы подтверждается шестью тестами:
\begin{lstlisting}
	TEST(Trapezoid_method_test, Test_function_1_n_10000);
\end{lstlisting}\par - взятие двумерного интеграла от функции $sin(x) * cos(y)$ по области $x : [0, 3]$, $y : [0, 3]$ c числом разбиений равным 10000. 
\begin{lstlisting}
	TEST(Trapezoid_method_test, Test_function_2_n_10000);
\end{lstlisting}\par - взятие двумерного интеграла от функции $x * y$ по области $x : [0, 3]$, $y : [0, 3]$ c числом разбиений равным 10000.
\begin{lstlisting}
    TEST(Trapezoid_method_test, Test_function_2_n_20000);
\end{lstlisting}\par - взятие двумерного интеграла от функции $x * y$ по области $x : [0, 3]$, $y : [0, 3]$ c числом разбиений равным 20000.
\begin{lstlisting}
    TEST(Trapezoid_method_test, Test_function_3_n_10000);
\end{lstlisting}\par - взятие двумерного интеграла от функции $pow(x, y)$ по области $x : [0, 3]$, $y : [0, 3]$ c числом разбиений равным 10000.
\begin{lstlisting}
    TEST(Trapezoid_method_test, Test_function_4_n_10000);
\end{lstlisting}\par - взятие двумерного интеграла от функции $x * x + y * y$ по области $x : [0, 3]$, $y : [0, 3]$ c числом разбиений равным 10000.
\begin{lstlisting}
    TEST(Trapezoid_method_test, Test_function_5_n_10000);
\end{lstlisting}\par - взятие двумерного интеграла от функции $1$ по области $x : [0, 3]$, $y : [0, 3]$ c числом разбиений равным 10000.
\par Все тесты проходят проверку, что является доказательством корректной работы программы.
\newpage

% Результаты экспериментов
\section*{Результаты экспериментов}
\addcontentsline{toc}{section}{Результаты экспериментов}
Эксперименты для оценки эффективности проводились на ПК со следующими характеристиками:

\begin{itemize}
\item Процессор: Intel Core i5-9300H, количество ядер: $4$, количество потоков: $8$, базовая (используемая в экспериментах) тактовая частота $2,40$ ГГц, Кэш-память: $8$ МБ
\item Оперативная память: $8$ ГБ (DDR4)
\item Операционная система: Microsoft Windows $10$ 
\end{itemize}

\par В данном эксперименте было вычислено время работы для последовательной и параллельной схемы нахождения многомерных определенных интегралов. А именно: для подынтегральной функции $x * y$ с пределами интегрирования $x : [0, 3]$, $y : [0, 3]$ и количеством отрезков разбиения $n=10000$ и $n=20000$. Помимо этого посмотрим на поведение времени работы схем для более сложной функции $x^y$, $n=10000$.
\par Результаты экспериментов представлены ниже.

\begin{table}[!h]
    \begin{tabular}{ | p{4cm} | p{4cm} | p{4cm} | p{4cm} | }
    \hline
    Кол-во\parпроцессов & Последовательный\parалгоритм, сек & Параллельный\parалгоритм, сек & Ускорение\\ \hline
    2    & 0.150869 & 0.0787042 & 1.91691 \\ \hline
    4    & 0.156857 & 0.0474068 & 3.30875\\ \hline
    8    & 0.153752 & 0.0301618 & 5.09758 \\ \hline
    9    & 0.153444 & 0.0504696 & 3.04032 \\ \hline
    16   & 0.155987 & 0.0567708 & 2.74766 \\ \hline
    \end{tabular}
    \caption{Результаты вычислительных экспериментов функции $x * y$ при разбиении $n=10000$}
\end{table}

\begin{table}[!h]
    \begin{tabular}{ | p{4cm} | p{4cm} | p{4cm} | p{4cm} | }
    \hline
    Кол-во\parпроцессов & Последовательный\parалгоритм, сек & Параллельный\parалгоритм, сек & Ускорение\\ \hline
    2    & 0.60464 & 0.306754 & 1.97109 \\ \hline
    4    & 0.60929 & 0.186803 & 3.26168\\ \hline
    8    & 0.612074 & 0.115482 & 5.3002 \\ \hline
    9    & 0.608349 & 0.133285 & 4.56427 \\ \hline
    16   & 0.617208 & 0.15709 & 3.92902 \\ \hline
    \end{tabular}
    \caption{Результаты вычислительных экспериментов функции $x * y$ при разбиении $n=10000$}
\end{table}
\newpage
\begin{table}[!h]
    \begin{tabular}{ | p{4cm} | p{4cm} | p{4cm} | p{4cm} | }
    \hline
    Кол-во\parпроцессов & Последовательный\parалгоритм, сек & Параллельный\parалгоритм, сек & Ускорение\\ \hline
    2    & 0.60464 & 0.306754 & 1.97109 \\ \hline
    4    & 2.95745 & 0.941694 & 3.14057\\ \hline
    8    & 2.96379 & 0.593524 & 4.99355 \\ \hline
    9    & 2.84777 & 0.605192 & 4.70556 \\ \hline
    16   & 2.91293 & 0.617306 & 4.51878 \\ \hline
    \end{tabular}
    \caption{Результаты вычислительных экспериментов более сложной функции $x^y$ при разбиении $n=10000$}
\end{table}
\par \textbf{Вывод:}
\par Как мы можем заметить при увеличении числа процессов действительно наблюдается ускорение работы алгоритма, его можно назвать существенным. И чем сложнее функция, тем больше ускорение параллельной схемы относительно последовательной. На восьми процессах можно наблюдать обратную картину. Ускорение постепенно уменьшается. Это должно быть связано с конфигурацией используемого процессора, а именно с числом потоков, их на данном процессоре - 8.
\par По мере увеличения числа процессов, ускорение будет уменьшаться и приближаться к цифрам меньше 1. Это связано с увеличением накладных расходов: происходит больше пересылок данных между процессами.
\newpage


% Заключение
\section*{Заключение}
\addcontentsline{toc}{section}{Заключение}
В ходе работы реализованы последовательная и параллельная версия алгоритма трапеций для вычисления многомерных интегралов. А также выполнены все поставленные подзадачи.
\par Для подтверждения корректности работы программы написаны тесты с использованием Google C++ Testing Framework.
\par Результаты проведенных экспериментов показывают, что параллельная реализация работает быстрее, чем последовательная. Также, ускорение времени работы параллельной схемы относительно последовательной, уменьшается достигнув числа процессов - 8 и уменьшается дальше вплоть до 1 и меньше.
\newpage


\begin{thebibliography}{1}
    \addcontentsline{toc}{section}{Список литературы}
    \bibitem{1} 
    Панасенко А.Г. Методические указания к решению задач по
    численному интегрированию : Учебно-методическое пособие /Калашников А.Л., Федоткин А.М., Фокина В.Н. - Нижний Новгород: Нижегородский госуниверситет, 2016. – 31 с.
    \bibitem{2}
    Воеводин В.В., Воеводин Вл.В. (2002). Параллельные вычисления. – СПб.: БХВ-Петербург.
    \bibitem{3} 
    Гергель В. П. Теория и практика параллельных вычислений. – 2007.
\end{thebibliography}
\newpage




\section*{Приложение}
\addcontentsline{toc}{section}{Приложение}
trapezoid\_method.h
\begin{lstlisting}
// Copyright 2021 Sablin Alexander

#ifndef MODULES_TASK_3_SABLIN_A_TRAPEZOID_METHOD_TRAPEZOID_METHOD_H_
#define MODULES_TASK_3_SABLIN_A_TRAPEZOID_METHOD_TRAPEZOID_METHOD_H_

double trapezoidMethodSequential(double (*f)(double, double), double a_x, double b_x,
    double a_y, double b_y, const int n);
double trapezoidMethodParallel(double (*f)(double, double), double a_x, double b_x,
    double a_y, double b_y, const int n);

#endif  //  MODULES_TASK_3_SABLIN_A_TRAPEZOID_METHOD_TRAPEZOID_METHOD_H_
\end{lstlisting}
trapezoid\_method.cpp
\begin{lstlisting}
// Copyright 2021 Sablin Alexander

#include <mpi.h>
#include "../../../modules/task_3/sablin_a_trapezoid_method/trapezoid_method.h"


double trapezoidMethodSequential(double (*f)(double, double), double a_x, double b_x,
    double a_y, double b_y, const int n) {
  double ans = 0;
  double x_h = (b_x - a_x) / n;
  double y_h = (b_y - a_y) / n;

  for (double x = a_x + x_h; x < b_x; x += x_h) {
      double cur_ans = (f(x, a_y) + f(x, b_y)) / 2;
      for (double y = a_y + y_h; y < b_y; y += y_h) {
          cur_ans += f(x, y);
      }
      ans += cur_ans;
  }

  ans += (f(a_x, a_y) + f(a_x, b_y) + f(b_x, a_y) + f(b_x, b_y)) / 4;
  for (double y = a_y + y_h; y < b_y; y += y_h) {
      ans += (f(a_x, y) + f(b_x, y)) / 2;
  }

  return ans * x_h * y_h;
}

double trapezoidMethodParallel(double (*f)(double, double), double a_x, double b_x,
    double a_y, double b_y, const int n) {
  int ProcRank, ProcNum;

  MPI_Comm_rank(MPI_COMM_WORLD, &ProcRank);
  MPI_Comm_size(MPI_COMM_WORLD, &ProcNum);
  double ans_all;
  double ans = 0.0;

  double h_x = (b_x - a_x) / n;
  double h_y = (b_y - a_y) / n;
  double x;

  x = a_x + h_x * (ProcRank + 1);

  while (x < b_x) {
      double cur_ans = (f(x, a_y) + f(x, b_y)) / 2;
      for (double y = a_y + h_y; y < b_y; y += h_y) {
          cur_ans += f(x, y);
      }
      ans += cur_ans;
      x += h_x * ProcNum;
  }
  MPI_Barrier(MPI_COMM_WORLD);
  MPI_Reduce(&ans, &ans_all, 1, MPI_DOUBLE, MPI_SUM, 0, MPI_COMM_WORLD);

  if (ProcRank == 0) {
      ans_all += (f(a_x, a_y) + f(a_x, b_y) + f(b_x, a_y) + f(b_x, b_y)) / 4;
      for (double y = a_y + h_y; y < b_y; y += h_y) {
          ans_all += (f(a_x, y) + f(b_x, y)) / 2;
      }
  }
  MPI_Barrier(MPI_COMM_WORLD);

  return ans_all * h_x * h_y;
}
\end{lstlisting}
main.cpp
\begin{lstlisting}
// Copyright 2021 Sablin Alexander

#include <gtest/gtest.h>
#include <cmath>
#include <gtest-mpi-listener.hpp>
#include "../../../modules/task_3/sablin_a_trapezoid_method/trapezoid_method.h"


double function_1(double x, double y) {
  return sin(x) * cos(y);
}

double function_2(double x, double y) {
  return x * y;
}

double function_3(double x, double y) {
  return pow(x, y);
}

double function_4(double x, double y) {
  return x * x + y * y;
}

double function_5(double x, double y) {
  return 1;
}

TEST(Trapezoid_method_test, Test_function_1_n_10000) {
    double (*f)(double, double) = function_1;
    const int n = 10000;
    double a_x = 0.0;
    double b_x = 3.0;
    double a_y = 0.0;
    double b_y = 3.0;
    int ProcRank;
    MPI_Comm_rank(MPI_COMM_WORLD, &ProcRank);
    double start, finish, parallel_time, sequential_time, parallel_res, sequential_res;
    if (ProcRank == 0) {
        start = MPI_Wtime();
    }
    MPI_Barrier(MPI_COMM_WORLD);
    parallel_res = trapezoidMethodParallel(f, a_x, b_x, a_y, b_y, n);
    MPI_Barrier(MPI_COMM_WORLD);
    if (ProcRank == 0) {
        finish = MPI_Wtime();
        parallel_time = finish - start;
        start = MPI_Wtime();
        sequential_res = trapezoidMethodSequential(f, a_x, b_x, a_y, b_y, n);
        finish = MPI_Wtime();
        sequential_time = finish - start;
        std::cout << "------------------------------------------------------" << std::endl;
        std::cout << "--- func : sin(x) * cos(y) " << std::endl;
        std::cout << "--- n : 10000 " << std::endl;
        std::cout << "--- Sequential time = " << sequential_time << std::endl;
        std::cout << "--- Parallel time = " << parallel_time << std::endl;
        std::cout << "--- Acceleration = " << sequential_time / parallel_time << std::endl;
        std::cout << "------------------------------------------------------" << std::endl;
        ASSERT_EQ(std::abs(parallel_res - sequential_res) < 0.015, true);
    }
}

TEST(Trapezoid_method_test, Test_function_2_n_10000) {
    double (*f)(double, double) = function_2;
    const int n = 10000;
    double a_x = 0.0;
    double b_x = 3.0;
    double a_y = 0.0;
    double b_y = 3.0;
    int ProcRank;
    MPI_Comm_rank(MPI_COMM_WORLD, &ProcRank);
    double start, finish, parallel_time, sequential_time, parallel_res, sequential_res;
    if (ProcRank == 0) {
        start = MPI_Wtime();
    }
    MPI_Barrier(MPI_COMM_WORLD);
    parallel_res = trapezoidMethodParallel(f, a_x, b_x, a_y, b_y, n);
    MPI_Barrier(MPI_COMM_WORLD);
    if (ProcRank == 0) {
        finish = MPI_Wtime();
        parallel_time = finish - start;
        start = MPI_Wtime();
        sequential_res = trapezoidMethodSequential(f, a_x, b_x, a_y, b_y, n);
        finish = MPI_Wtime();
        sequential_time = finish - start;
        std::cout << "------------------------------------------------------" << std::endl;
        std::cout << "--- func : x * y " << std::endl;
        std::cout << "--- n : 10000 " << std::endl;
        std::cout << "--- Sequential time = " << sequential_time << std::endl;
        std::cout << "--- Parallel time = " << parallel_time << std::endl;
        std::cout << "--- Acceleration = " << sequential_time / parallel_time << std::endl;
        std::cout << "------------------------------------------------------" << std::endl;
        ASSERT_EQ(std::abs(parallel_res - sequential_res) < 0.015, true);
    }
}

TEST(Trapezoid_method_test, Test_function_2_n_20000) {
    double (*f)(double, double) = function_2;
    const int n = 20000;
    double a_x = 0.0;
    double b_x = 3.0;
    double a_y = 0.0;
    double b_y = 3.0;
    int ProcRank;
    MPI_Comm_rank(MPI_COMM_WORLD, &ProcRank);
    double start, finish, parallel_time, sequential_time, parallel_res, sequential_res;
    if (ProcRank == 0) {
        start = MPI_Wtime();
    }
    MPI_Barrier(MPI_COMM_WORLD);
    parallel_res = trapezoidMethodParallel(f, a_x, b_x, a_y, b_y, n);
    MPI_Barrier(MPI_COMM_WORLD);
    if (ProcRank == 0) {
        finish = MPI_Wtime();
        parallel_time = finish - start;
        start = MPI_Wtime();
        sequential_res = trapezoidMethodSequential(f, a_x, b_x, a_y, b_y, n);
        finish = MPI_Wtime();
        sequential_time = finish - start;
        std::cout << "------------------------------------------------------" << std::endl;
        std::cout << "--- func : x * y " << std::endl;
        std::cout << "--- n : 20000 " << std::endl;
        std::cout << "--- Sequential time = " << sequential_time << std::endl;
        std::cout << "--- Parallel time = " << parallel_time << std::endl;
        std::cout << "--- Acceleration = " << sequential_time / parallel_time << std::endl;
        std::cout << "------------------------------------------------------" << std::endl;
        ASSERT_EQ(std::abs(parallel_res - sequential_res) < 0.015, true);
    }
}

TEST(Trapezoid_method_test, Test_function_3_n_10000) {
    double (*f)(double, double) = function_3;
    const int n = 10000;
    double a_x = 0.0;
    double b_x = 3.0;
    double a_y = 0.0;
    double b_y = 3.0;
    int ProcRank;
    MPI_Comm_rank(MPI_COMM_WORLD, &ProcRank);
    double start, finish, parallel_time, sequential_time, parallel_res, sequential_res;
    if (ProcRank == 0) {
        start = MPI_Wtime();
    }
    MPI_Barrier(MPI_COMM_WORLD);
    parallel_res = trapezoidMethodParallel(f, a_x, b_x, a_y, b_y, n);
    MPI_Barrier(MPI_COMM_WORLD);
    if (ProcRank == 0) {
        finish = MPI_Wtime();
        parallel_time = finish - start;
        start = MPI_Wtime();
        sequential_res = trapezoidMethodSequential(f, a_x, b_x, a_y, b_y, n);
        finish = MPI_Wtime();
        sequential_time = finish - start;
        std::cout << "------------------------------------------------------" << std::endl;
        std::cout << "--- func : pow(x, y) " << std::endl;
        std::cout << "--- n : 10000 " << std::endl;
        std::cout << "--- Sequential time = " << sequential_time << std::endl;
        std::cout << "--- Parallel time = " << parallel_time << std::endl;
        std::cout << "--- Acceleration = " << sequential_time / parallel_time << std::endl;
        std::cout << "------------------------------------------------------" << std::endl;
        ASSERT_EQ(std::abs(parallel_res - sequential_res) < 0.015, true);
    }
}

TEST(Trapezoid_method_test, Test_function_4_n_10000) {
    double (*f)(double, double) = function_4;
    const int n = 10000;
    double a_x = 0.0;
    double b_x = 3.0;
    double a_y = 0.0;
    double b_y = 3.0;
    int ProcRank;
    MPI_Comm_rank(MPI_COMM_WORLD, &ProcRank);
    double start, finish, parallel_time, sequential_time, parallel_res, sequential_res;
    if (ProcRank == 0) {
        start = MPI_Wtime();
    }
    MPI_Barrier(MPI_COMM_WORLD);
    parallel_res = trapezoidMethodParallel(f, a_x, b_x, a_y, b_y, n);
    MPI_Barrier(MPI_COMM_WORLD);
    if (ProcRank == 0) {
        finish = MPI_Wtime();
        parallel_time = finish - start;
        start = MPI_Wtime();
        sequential_res = trapezoidMethodSequential(f, a_x, b_x, a_y, b_y, n);
        finish = MPI_Wtime();
        sequential_time = finish - start;
        std::cout << "------------------------------------------------------" << std::endl;
        std::cout << "--- func : x * x + y * y " << std::endl;
        std::cout << "--- n : 10000 " << std::endl;
        std::cout << "--- Sequential time = " << sequential_time << std::endl;
        std::cout << "--- Parallel time = " << parallel_time << std::endl;
        std::cout << "--- Acceleration = " << sequential_time / parallel_time << std::endl;
        std::cout << "------------------------------------------------------" << std::endl;
        ASSERT_EQ(std::abs(parallel_res - sequential_res) < 0.015, true);
    }
}

TEST(Trapezoid_method_test, Test_function_5_n_10000) {
    double (*f)(double, double) = function_5;
    const int n = 10000;
    double a_x = 0.0;
    double b_x = 3.0;
    double a_y = 0.0;
    double b_y = 3.0;
    int ProcRank;
    MPI_Comm_rank(MPI_COMM_WORLD, &ProcRank);
    double start, finish, parallel_time, sequential_time, parallel_res, sequential_res;
    if (ProcRank == 0) {
        start = MPI_Wtime();
    }
    MPI_Barrier(MPI_COMM_WORLD);
    parallel_res = trapezoidMethodParallel(f, a_x, b_x, a_y, b_y, n);
    MPI_Barrier(MPI_COMM_WORLD);
    if (ProcRank == 0) {
        finish = MPI_Wtime();
        parallel_time = finish - start;
        start = MPI_Wtime();
        sequential_res = trapezoidMethodSequential(f, a_x, b_x, a_y, b_y, n);
        finish = MPI_Wtime();
        sequential_time = finish - start;
        std::cout << "------------------------------------------------------" << std::endl;
        std::cout << "--- func : 1 " << std::endl;
        std::cout << "--- n : 10000 " << std::endl;
        std::cout << "--- Sequential time = " << sequential_time << std::endl;
        std::cout << "--- Parallel time = " << parallel_time << std::endl;
        std::cout << "--- Acceleration = " << sequential_time / parallel_time << std::endl;
        std::cout << "------------------------------------------------------" << std::endl;
        ASSERT_EQ(std::abs(parallel_res - sequential_res) < 0.015, true);
    }
}

int main(int argc, char** argv) {
  ::testing::InitGoogleTest(&argc, argv);
  MPI_Init(&argc, &argv);

  ::testing::AddGlobalTestEnvironment(new GTestMPIListener::MPIEnvironment);
  ::testing::TestEventListeners& listeners =::testing::UnitTest::GetInstance()->listeners();

  listeners.Release(listeners.default_result_printer());
  listeners.Release(listeners.default_xml_generator());

  listeners.Append(new GTestMPIListener::MPIMinimalistPrinter);

  return RUN_ALL_TESTS();
}
\end{lstlisting}
\end{document}