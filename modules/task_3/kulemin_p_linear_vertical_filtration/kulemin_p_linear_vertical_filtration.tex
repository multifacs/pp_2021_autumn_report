\documentclass{report}

\usepackage[warn]{mathtext}
\usepackage[T2A]{fontenc}
\usepackage[utf8]{luainputenc}
\usepackage[english, russian]{babel}
\usepackage[pdftex]{hyperref}
\usepackage[12pt]{extsizes}
\usepackage{listings}
\usepackage{color}
\usepackage{geometry}
\usepackage{enumitem}
\usepackage{multirow}
\usepackage{graphicx}
\usepackage{indentfirst}
\usepackage{amsmath}

\geometry{a4paper,top=2cm,bottom=2cm,left=2.5cm,right=1.5cm}
\setlength{\parskip}{0.5cm}
\setlist{nolistsep, itemsep=0.3cm,parsep=0pt}

\usepackage{listings}
\lstset{language=C++,
        basicstyle=\footnotesize,
		keywordstyle=\color{blue}\ttfamily,
		stringstyle=\color{red}\ttfamily,
		commentstyle=\color{green}\ttfamily,
		morecomment=[l][\color{red}]{\#}, 
		tabsize=4,
		breaklines=true,
  		breakatwhitespace=true,
  		title=\lstname,       
}

\makeatletter
\renewcommand\@biblabel[1]{#1.\hfil}
\makeatother

\begin{document}

\begin{titlepage}

\begin{center}
Министерство науки и высшего образования Российской Федерации
\end{center}

\begin{center}
Федеральное государственное автономное образовательное учреждение высшего образования \\
Национальный исследовательский Нижегородский государственный университет им. Н.И. Лобачевского
\end{center}

\begin{center}
Институт информационных технологий, математики и механики
\end{center}

\vspace{4em}

\begin{center}
\textbf{\LargeОтчет по лабораторной работе} \\
\end{center}
\begin{center}
\textbf{\Large«Линейная фильтрация изображений (блочное разбиение). Ядро Гаусса 3x3»} \\
\end{center}

\vspace{4em}

\newbox{\lbox}
\savebox{\lbox}{\hbox{text}}
\newlength{\maxl}
\setlength{\maxl}{\wd\lbox}
\hfill\parbox{7cm}{
\hspace*{5cm}\hspace*{-5cm}\textbf{Выполнил:} \\ студент группы 381906-2 \\ Кулемин П.А\\
\\
\hspace*{5cm}\hspace*{-5cm}\textbf{Проверил:}\\ доцент кафедры МОСТ, \\ кандидат технических наук \\ Сысоев А. В.\\
}
\vspace{\fill}

\begin{center} Нижний Новгород \\ 2021 \end{center}

\end{titlepage}

\setcounter{page}{2}

% Содержание
\tableofcontents
\newpage

% Введение
\section*{Введение}
\addcontentsline{toc}{section}{Введение}
\par Фильтрация изображений - это подавление шума целевого изображения при максимально возможном сохранении деталей изображения. Это неотъемлемая часть предварительной обработки изображения. Качество обработки напрямую влияет на эффект последующей обработки и надежность анализа. Матричные фильтры в науке нужны для подготовки изображений; улучшения из визуального восприятия; распознавания образов – компьютерное зрение.Размытие по Гауссу в цифровой обработке изображений — способ размытия изображения с помощью функции Гаусса, названной в честь немецкого математика Карла Фридриха Гаусса.
Этот эффект широко используется в графических редакторах для уменьшения шума изображения и снижения детализации. 
\newpage

% Постановка задачи
\section*{Постановка задачи}
\addcontentsline{toc}{section}{Постановка задачи}
\par В данном лабораторной работе требуется реализовать последовательный и параллельный алгоритмы линейной фильтрации изображений с блочным разбиением с ядром Гаусса 3х3, провести серию экспериментв для сравнения времени работы алгоритмов, на основе которых сделать выводы о эффективности программы, используя при этом фрэймворк для разработки автоматических тестов Google Test.
\par Параллельный алгоритм должен быть реализован при помощи технологии MPI.
\newpage

% Описание алгоритма
\section*{Описание алгоритма}
\addcontentsline{toc}{section}{Описание алгоритма}
\par 
Создаем ядро Гаусса размером 3х3. 
\par Агоритм Фильтра Гаусса :
\begin{itemize}
\item Для каждого пикселя исходного изображения рассматривается область 3x3. 
\item Значение каждого пикселя из этой области умножается на значение из соответствующей ячейки в ядре , и полученный результат от произведения прибавляется к переменой color. 
\item Для нового изображения соответствующий пиксель кладётся значение специально созданной переменной color.
\end{itemize}

\newpage

% Схема распараллеливания
\section*{Схема распараллеливания}
\addcontentsline{toc}{section}{Схема распараллеливания}
\par 
Для распараллеливания алгоритма фильтра Гаусса необходимо разделить массив значений исходного изображения блочно на подмассивы
\begin{itemize}
\item Число разбиений равно числу процессов 
\item Изображение делим на прямоугольники. Но изображение хранится в однородном массиве, то необходимо каждый раз создавать однородный массив прямоугольника специальной функцией.
\item К каждому прямоугольнику разбиения применить фильтр Гаусса.
\item "Сливаем" итоговое изображение
\end{itemize}
\newpage

% Описание программной реализации
\section*{Описание программной реализации}
\addcontentsline{toc}{section}{Описание программной реализации}
Программа состоит из заголовочного файла linear\_vertical\_filtration.h и двух файлов исходного кода linear\_vertical\_filtration.cpp и main.cpp.
\par В заголовочном файле находятся прототипы функций для последовательного и параллельного алгоритмов линейной фильтрации изображений .
\par Функция для последовательного алгоритма:
\begin{lstlisting}
std::vector<float> getSequentialOperations(const std::vector<float>& matrix,
 const std::vector<float>& img, int weight, int height);
\end{lstlisting}
Первый параметр функции является ссылка на ядро, второй - ссылка на изображение, третий - ширина изображения, четвертый - высота изображения.
\par Функция для параллельного
алгоритма:
\begin{lstlisting}
std::vector<float> getParallelOperations(const std::vector<float>& matrix,
const std::vector<float>& img, int weight, int height) 
\end{lstlisting}
Входные параметры данной функции совпадают с входными параметрами функции для последовательного алгоритма.
\par В файле исходного кода linear\_vertical\_filtration.cpp содержится реализация функций, объявленных в заголовочном файле. В файле исходного кода main.cpp содержатся тесты для проверки корректности выполения программы.
\newpage

% Подтверждение корректности
\section*{Подтверждение корректности}
\addcontentsline{toc}{section}{Подтверждение корректности}
Для подтверждения корректности работы данной программы были разработаны тесты с помощью фрэймфорка Google Test. В каждом из тестов изображение обрабатывается параллельно, затем последовательно. после чего чего результаты сравниваются.

\par Успешное прохождение всех тестов подтверждает корректность работы программы.
\newpage

% Результаты экспериментов
\section*{Результаты экспериментов}
\addcontentsline{toc}{section}{Результаты экспериментов}
Вычислительные эксперименты для оценки эффективности работы параллельного алгоритма проводились на ПК со следующими характеристиками:
\begin{itemize}
\item Процессор: Intel Core i5 , 2.5 ГГц (3.5 ГГц, в режиме Turbo), количество ядер: 4;
\item Оперативная память: 8 ГБ (DDR4), 3200 МГц;
\item Операционная система: Windows 10 .
\end{itemize}

\par Эксперименты проводились на различном числе процессов для обработки изображения .
\par Результаты экспериментов представлены в Таблице 1.

\begin{table}[!h]
\caption{Результаты вычислительных экспериментов в тесте 4}
\centering
\begin{tabular}{| p{2cm} | p{3cm} | p{4cm} | p{2cm} |}
\hline
Количество процессов & Время работы последовательного алгоритма (в секундах) & Время работы параллельного алгоритма (в секундах) & Ускорение  \\[5pt]
\hline
1        & 3.3475        & 3.76676     & 0.888696       \\
2        & 3.35775        & 2.3388     & 1.43567       \\
3        & 3.34642        & 1.89855     & 1.76261       \\
4        & 3.35924        & 1.68735     & 1.99084       \\
5        & 3.34809        & 1.57535     & 2.1253       \\
6        & 3.35504        & 1.51344     & 2.21683       \\
7        & 3.39476        & 1.51552     & 2.24       \\
9        & 3.37919        & 1.5558     & 2.17199	    \\
11        & 3.41202        & 1.66682     & 2.04703	  \\
33        & 3.37404        & 1.99409     & 1.69202	  \\
\hline
\end{tabular}
\end{table}



\newpage

% Заключение
\section*{Заключение}
\addcontentsline{toc}{section}{Заключение}
В ходе выполнения  данной лабораторной работы были разработаны последовательный и параллельный алгоритмы  линейной фильтрации изображений с ядром Гаусса 3х3 .
\par Проведенные тесты показали корректность реализованной программы, а  эксперименты доказали, что параллельная версия алгоритма работает быстрее, а значит задача параллельного программирования решена успешно.
\newpage

% Литература
\section*{Литература}
\addcontentsline{toc}{section}{Литература}
\begin{enumerate}

\item Гергель В. П. Теория и практика параллельных вычислений. – 2007. 
\item Википедия - Электронный ресурс. URL: https://ru.wikipedia.org/wiki/%D0%A0%D0%B0%D0%B7%D0%BC%D1%8B%D1%82%D0%B8%D0%B5_%D0%BF%D0%BE_%D0%93%D0%B0%D1%83%D1%81%D1%81%D1%83
\itemГергель В. П., Стронгин Р. Г. Основы параллельных вычислений для многопроцессорных вычислительных систем. – 2003.
\end{enumerate} 
\newpage

% Приложение
\section*{Приложение}
\addcontentsline{toc}{section}{Приложение}

linear\_vertical\_filtration.cpp 
\begin{lstlisting}
#include <mpi.h>
#include <vector>
#include <random>
#include <ctime>
#include <iostream>
#include "../../../modules/task_3/kulemin_p_linear_vertical_filtration/linear_vectrical_filtration.h"

using std::vector;


void getKernell(vector<float>* matrix, float sigma) {
    std::mt19937 gen(time(0));
    std::uniform_real_distribution<> urd(0, 1);
    float norm = 0;
    matrix->resize(9);
    for (int x = -1; x <= 1; x++) {
        for (int y = -1; y <= 1; y++) {
            int idx = (x + 1) * 3 + (y + 1);
            (*matrix)[idx] = std::exp(-(x * x + y * y) / (sigma * sigma));
            norm += (*matrix)[idx];
        }
    }
    for (int i = 0; i < 9; i++) {
        if ((i != 1) && (i != 4) && (i != 7))(*matrix)[i] = 0.f;
        (*matrix)[i] /= norm;
    }
}

int calculate_index(int x, int y, int weight) {
    return y * weight + x;
}


float calculatedNewPixelColor(const vector<float>& matrix,
 const vector<float>&image, int height,
    int weight, const int x, const int y) {
    float color = 0;
    for (int i = -1; i <= 1; i++) {
        for (int j = -1; j <= 1; j++) {
            int idx = (i + 1) * 3 + j + 1;
            float imgColor = image[calculate_index(clamp(x + j, weight - 1, 0),
             clamp(y + i, height - 1, 0), weight)];
            color += (imgColor) * matrix[idx];
        }
    }
    return color;
}


void getRandomImg(std::vector<float>* img, int weight, int height) {
    std::mt19937 gen(time(0));
    std::uniform_real_distribution<> urd(0, 1);

    img->resize(weight*height);
    for (int i = 0; i < (weight*height); ++i) {
        img->at(i) = urd(gen);
    }
}

std::vector<float> getSequentialOperations(const std::vector<float>& matrix,
 const std::vector<float>& img, int weight, int height) {
    vector<float> resultMatrix(weight * height);

    for (int x = 0; x < weight; ++x) {
        for (int  y = 0; y < height; y++) {
            float value = calculatedNewPixelColor(matrix, img,
             height, weight, x, y);
            resultMatrix[calculate_index(x, y, weight)] = value;
        }
    }

    return resultMatrix;
}

std::vector<float> getParallelOperations(const std::vector<float>& matrix,
const std::vector<float>& img, int weight, int height) {
    int num, rank;
    MPI_Comm_size(MPI_COMM_WORLD, &num);
    MPI_Comm_rank(MPI_COMM_WORLD, &rank);
    int delta = weight / num;
    int rem = weight % num;
    vector<int> processes(num);
    vector<int> toright(num);
    vector< vector<float> > buffer(height);
    vector<float>res;
    for (int i = 0; i < height; i++)
    buffer[i].resize(weight);
    processes[0] = delta + rem;
    toright[0] = 0;
    for (int i = 1; i < num; i++) {
        processes[i] = delta;
        toright[i] = toright[i - 1] + processes[i-1];
    }
    if (delta) {
            for (int row = 0; row < (height); row++) {
               MPI_Scatterv(img.data() + row * weight, processes.data(),
               toright.data(), MPI_FLOAT, (buffer[row]).data(), processes[rank],
               MPI_FLOAT, 0, MPI_COMM_WORLD);
            }


        vector<float>ipart;

        for (int i = 0; i < height; i++) {
            for (int j = 0; j < processes[rank]; j++) {
                ipart.push_back(buffer[i][j]);
            }
        }
        ipart = getSequentialOperations(matrix, ipart, processes[rank], height);

        for (int i = 0; i < height; i++) {
            MPI_Gatherv(ipart.data() + i * processes[rank],
            processes[rank], MPI_FLOAT,
            (buffer[i]).data(), processes.data(),
            toright.data(), MPI_FLOAT, 0, MPI_COMM_WORLD);
        }
        if (rank == 0) {
            for (int i = 0; i < height; i++) {
                for (int j = 0; j < weight; j++) {
                    res.push_back(buffer[i][j]);
                }
            }
        }
    } else {
        if (rank == 0) {
            res = getSequentialOperations(matrix, img, weight, height);
        }
    }
    return res;
}
\end{lstlisting}
main.cpp
\begin{lstlisting}
// Copyright 2021 Zaytsev Mikhail
#include <gtest/gtest.h>
#include <vector>
#include "./linear_vectrical_filtration.h"
#include <gtest-mpi-listener.hpp>

TEST(Parallel_Matrix_Multiplacition, mRows_Eq_mColumns_50) {
    int currentProcess;
    MPI_Comm_rank(MPI_COMM_WORLD, &currentProcess);

    std::vector<float> matrix, img;
    int height = 50;
    int weight = 50;
    getKernell(&matrix);

    if (currentProcess == 0) {
        getRandomImg(&img, weight, height);
    }
    std::vector<float> globalMatrix = getParallelOperations(matrix, img,
   weight, height);
    if (currentProcess == 0) {
        std::vector<float> referenceMatrix = getSequentialOperations(matrix,
        img, weight, height);
        ASSERT_EQ(globalMatrix, referenceMatrix);
    }
}

TEST(Parallel_Matrix_Multiplacition, mRows_Eq_mColumns_5) {
    int currentProcess;
    MPI_Comm_rank(MPI_COMM_WORLD, &currentProcess);

    std::vector<float> matrix, img;
    int height = 5;
    int weight = 5;
    getKernell(&matrix);

    if (currentProcess == 0) {
        getRandomImg(&img, weight, height);
    }

    std::vector<float> globalMatrix = getParallelOperations(matrix,
    img, weight, height);

    if (currentProcess == 0) {
        std::vector<float> referenceMatrix = getSequentialOperations(matrix,
        img, weight, height);
        ASSERT_EQ(globalMatrix, referenceMatrix);
    }
}

TEST(Parallel_Matrix_Multiplacition, mRows_Eq_mColumns_211) {
    int currentProcess;
    MPI_Comm_rank(MPI_COMM_WORLD, &currentProcess);

    std::vector<float> matrix, img;
    int height = 211;
    int weight = 211;
    getKernell(&matrix);

    if (currentProcess == 0) {
        getRandomImg(&img, weight, height);
    }

    std::vector<float> globalMatrix = getParallelOperations(matrix,
    img, weight, height);

    if (currentProcess == 0) {
        std::vector<float> referenceMatrix = getSequentialOperations(matrix,
        img, weight, height);
        ASSERT_EQ(globalMatrix, referenceMatrix);
    }
}

TEST(Parallel_Matrix_Multiplacition, mRows_Gr_mColumns_150_100) {
    int currentProcess;
    MPI_Comm_rank(MPI_COMM_WORLD, &currentProcess);
    std::vector<float> matrix, img;
    int height = 150;
    int weight = 100;
    getKernell(&matrix);
    if (currentProcess == 0) {
        getRandomImg(&img, weight, height);
    }

    std::vector<float> globalMatrix = getParallelOperations(matrix,
    img, weight, height);

    if (currentProcess == 0) {
        std::vector<float> referenceMatrix = getSequentialOperations(matrix,
        img, weight, height);
        ASSERT_EQ(globalMatrix, referenceMatrix);
    }
}

TEST(Parallel_Matrix_Multiplacition, mRows_Le_mColumns_100_150) {
    int currentProcess;
    MPI_Comm_rank(MPI_COMM_WORLD, &currentProcess);

    std::vector<float> matrix, img;
    int height = 100;
    int weight = 150;
    getKernell(&matrix);
    if (currentProcess == 0) {
        getRandomImg(&img, weight, height);
    }

    std::vector<float> globalMatrix = getParallelOperations(matrix,
    img, weight, height);

    if (currentProcess == 0) {
        std::vector<float> referenceMatrix = getSequentialOperations(matrix,
        img, weight, height);
        ASSERT_EQ(globalMatrix, referenceMatrix);
    }
}

int main(int argc, char** argv) {
    ::testing::InitGoogleTest(&argc, argv);
    MPI_Init(&argc, &argv);

    ::testing::AddGlobalTestEnvironment(new GTestMPIListener::MPIEnvironment);
    ::testing::TestEventListeners& listeners =
        ::testing::UnitTest::GetInstance()->listeners();

    listeners.Release(listeners.default_result_printer());
    listeners.Release(listeners.default_xml_generator());

    listeners.Append(new GTestMPIListener::MPIMinimalistPrinter);
    return RUN_ALL_TESTS();
}
\end{lstlisting}

\end{document}