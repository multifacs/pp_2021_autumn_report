\documentclass{report}

\usepackage[warn]{mathtext}
\usepackage[T2A]{fontenc}
\usepackage[utf8]{luainputenc}
\usepackage[english, russian]{babel}
\usepackage[pdftex]{hyperref}
\usepackage[12pt]{extsizes}
\usepackage{listings}
\usepackage{color}
\usepackage{geometry}
\usepackage{enumitem}
\usepackage{multirow}
\usepackage{graphicx}
\usepackage{indentfirst}
\usepackage{amsmath}

\geometry{a4paper,top=2cm,bottom=2cm,left=2.5cm,right=1.5cm}
\setlength{\parskip}{0.5cm}
\setlist{nolistsep, itemsep=0.3cm,parsep=0pt}

\usepackage{listings}
\lstset{language=C++,
        basicstyle=\footnotesize,
		keywordstyle=\color{blue}\ttfamily,
		stringstyle=\color{red}\ttfamily,
		commentstyle=\color{green}\ttfamily,
		morecomment=[l][\color{red}]{\#}, 
		tabsize=4,
		breaklines=true,
  		breakatwhitespace=true,
  		title=\lstname,       
}

\makeatletter
\renewcommand\@biblabel[1]{#1.\hfil}
\makeatother

\begin{document}

\begin{titlepage}

\begin{center}
Министерство науки и высшего образования Российской Федерации
\end{center}

\begin{center}
Федеральное государственное автономное образовательное учреждение высшего образования \\
Национальный исследовательский Нижегородский государственный университет им. Н.И. Лобачевского
\end{center}

\begin{center}
Институт информационных технологий, математики и механики
\end{center}

\vspace{4em}

\begin{center}
\textbf{\LargeОтчет по лабораторной работе} \\
\end{center}
\begin{center}
\textbf{\Large«Вычисление многомерных интегралов с использованием многошаговой схемы (метод прямоугольников»} \\
\end{center}

\vspace{4em}

\newbox{\lbox}
\savebox{\lbox}{\hbox{text}}
\newlength{\maxl}
\setlength{\maxl}{\wd\lbox}
\hfill\parbox{7cm}{
\hspace*{5cm}\hspace*{-5cm}\textbf{Выполнила:} \\ студент группы 381906-2 \\ Гордей М.В\\
\\
\hspace*{5cm}\hspace*{-5cm}\textbf{Проверил:}\\ доцент кафедры МОСТ, \\ кандидат технических наук \\ Сысоев А. В.\\
}
\vspace{\fill}

\begin{center} Нижний Новгород \\ 2021 \end{center}

\end{titlepage}

\setcounter{page}{2}

% Содержание
\tableofcontents
\newpage

% Введение
\section*{Введение}
\addcontentsline{toc}{section}{Введение}
\par В лабораторной работе рассматривается известная математическая задача–вычисление значений определенных интегралов.
Задача вычисления определенного интеграла I для некоторой заданной на отрезке [a, b] функции f(x) является классической задачей математического анализа. Известно, что для функций, имеющих на [a, b] конечное число точек разрыва первого рода, такое значение существует, единственно и может быть формально получено по определению как\\
\begin{equation}\iiint{f(x_1,\dots,x_k) dx_1 \dots dx_k} = h_i\sum_{i=1}^{N}{f(p_{i})} \end{equation}
В данной лабораторной работе изучаются подходы к распараллеливанию метода прямоугольников,идеи по алгоритмической оптимизации, приводящие к уменьшению времени вычислений, вопросы параллельной отладки.
\newpage

% Постановка задачи
\section*{Постановка задачи}
\addcontentsline{toc}{section}{Постановка задачи}
\par В данной лабораторной работе требуется реализовать последовательный и параллельный алгоритмы вычисления многомерных интегралов методом прямоугольников, проверить корректность работы программы,провести серию экспериментов по оценке времени работы программы.

\par Параллельный алгоритм должен быть реализован при помощи технологии MPI.Для проверки корректности работы алгоритмов будут использзоваться Google C++ Testing Framework.
\newpage

% Описание алгоритма
\section*{Описание алгоритма}
\addcontentsline{toc}{section}{Описание алгоритма}
\par \addcontentsline{toc}{section}{Метод решения}
Алгоритм вычисления интегралов методом прямоугольников:
\parПусть функция y = f(x) непрерывна на отрезке [a; b]. Нам требуется вычислить определенный интеграл.

\par1) Весь участок [a, b] делим на n равных частей с шагом h = (b - a)/ n.
\par2) Определяем значение $y_i$ подынтегральной функции $f(x)$ в каждой части деления, т.е $y_i$=f($x_i$), $i\in[0, n]$.
\par3) В каждой части деления подынтегральную функцию f(x) аппроксимируем интерполяционным многочленом степени n = 0
\par4) Для каждой части деления определяем площадь S{i} частичного прямоугольника.
\par5) Суммируем эти площади. Приближенное значение интеграла I равно сумме площадей частичных прямоугольников.

\newpage

% Схема распараллеливания
\section*{Схема распараллеливания}
\addcontentsline{toc}{section}{Схема распараллеливания}
\par
Основная идея реализации параллельного  алгоритма состоит в параллельном вычислении локальных сумм по формуле средних прямоугольников. 
Заданное количество интервалов распределяется между определенным количеством процессов. Каждый процесс вычисляет значения подынтегральной функции для определенного числа отрезков, суммирует эти значения и отправляет получившиеся локальные суммы в процесс с рангом 0. В 0-ом процессе все локальные суммы сначала складываются в глобальную сумму. После результат будет умножаться на шаги разбиения отрезков интегрирования.В итоге получится приближенное значение искомого интеграла.

\newpage

% Описание программной реализации
\section*{Описание программной реализации}
\addcontentsline{toc}{section}{Описание программной реализации}
Программа состоит из заголовочного файла riemann\_sum.h и двух файлов исходного кода riemann\_sum.cpp и main.cpp.
\par В заголовочном файле находятся прототипы функций для последовательного и параллельного вычисления многомерных интегралов.
\par Функция для последовательного алгоритма:
\begin{lstlisting}
double getSequentialIntegrals(
    const int n, const std::vector<std::pair<double, double>>& limits,
    const std::function<double(std::vector<double>)>& f) 
\end{lstlisting}
В качестве входных  параметров : число отрезков разбиения, массив пределов  и подынтегральная функция.
\par Функция для параллельного
алгоритма:
\begin{lstlisting}
double getParallelIntegrals(
    const int n, const std::vector<std::pair<double, double>>& limits,
    const std::function<double(std::vector<double>)>& f)
\end{lstlisting}
Входные параметры данной функции совпадают с входными параметрами функции для последовательного алгоритма.
\par В файле исходного кода  riemann\textunderscore sum.cpp содержится реализация функций, объявленных в заголовочном файле. В файле исходного кода main.cpp содержатся тесты для проверки корректности программы.
\newpage

% Подтверждение корректности
\section*{Подтверждение корректности}
\addcontentsline{toc}{section}{Подтверждение корректности}
Для подтверждения корректности работы данной программы я разработала  тесты с помощью фрэймфорка Google Test. В каждом из тестов вычисляется значение интеграла заданной размерности при помощи последовательного и параллельного алгоритмов, подсчитывается время работы обоих алгоритмов, находится ускорение и затем результаты вычисления интеграла последовательным и параллельным способом сравниваются между собой в точности до эпсилон.

\par Успешное прохождение всех тестов подтверждает корректность работы программы.
\newpage

% Результаты экспериментов
\section*{Результаты экспериментов}
\addcontentsline{toc}{section}{Результаты экспериментов}
Вычислительные эксперименты для оценки эффективности работы параллельного алгоритма проводились на ПК со следующими характеристиками:
\begin{itemize}
\item Процессор:Intel(R) Core(TM) i5-8265U 1.8GHz, количество ядер: 4; логических процессов:8;
\item Оперативная память: 8 ГБ
\item Операционная система: Windows 10 Home.
\end{itemize}

\par Эксперименты проводились на различном числе процессов для вычисления двумерного  интеграла  с количеством разбиений, равным 200. 
\par Результаты экспериментов представлены в Таблице 1.

\begin{table}[!h]
\caption{Результаты вычислительных экспериментов в тесте 4}
\centering
\begin{tabular}{| p{2cm} | p{3cm} | p{4cm} | p{2cm} |}
\hline
Количество процессов & Время работы последовательного алгоритма (в секундах) & Время работы параллельного алгоритма (в секундах) & Ускорение  \\[5pt]
\hline
1        & 0.06461        & 0.063504    &1.017449       \\
2        & 0.0655      &0.032143   &  2.038102     \\
3        &  0.0843121      & 0.024497      &3.441802 \\
4        &  0.0818733       &  0.019922     & 4.109631     \\
5        &  0.0826417      & 0.025347   &   3.260388     \\
6        &  0.108253      &   0.012125   &    8.928254    \\
7        &    0.106907      & 0.014567      &  7.338798     \\
8        &   0.123136       &   0.016219  &   7.592077   \\
10       &    0.159396       &   0.012955  &   12.303798   \\
\hline
\end{tabular}
\end{table}

\par По данным, полученным в результате экспериментов, можно сделать вывод о том, что параллельный алгоритм работает  быстрее, чем последовательный. Процессов имеет 8 логических ядер, максимальное ускорение параллельного алгоритма  по сравнению с последовательной реализацией составляет 8 раз. В эксперименте на 10 процессов результат ускорения превосходит это значение. Это связано с разной алгоритмической сложностью реализаций из-за использования MPI функций. Сложность параллельного алгоритма меньше последовательного, также результаты работы программы кэшируются.
\newpage


% Заключение
\section*{Заключение}
\addcontentsline{toc}{section}{Заключение}
В результате выполнения   данной лабораторной работы были разработаны последовательный и параллельный алгоритмы вычисления многомерных интегралов методом прямоугольников. Проведенные тесты показали корректность реализованной программы, а проведенные эксперименты доказали, что параллельная версия алгоритма работает быстрее.
\newpage

% Литература
\section*{Литература}
\addcontentsline{toc}{section}{Литература}
\begin{enumerate}
\item Антонов А.С. Параллельное программирование с использованием технологии MPI:Учебное пособие.-М.: Изд-во МГУ, 2004.-71 с.
\item Козинов Е.А., Мееров И.Б., Сысоев А.В. «Лабораторная работа №2. Вычисление определенного интеграла». Нижний Новгород, 2010, 38 с.


\end{enumerate} 
\newpage

% Приложение
\section*{Приложение}
\addcontentsline{toc}{section}{Приложение}

riemann\_sum.h
\begin{lstlisting}
// Copyright 2021 Gordey Maria
#include "../../../modules/task_3/gordey_m_riemann_sum/riemann_sum.h"

#include <mpi.h>

double getSequentialIntegrals(
    const int n, const std::vector<std::pair<double, double>>& limits,
    const std::function<double(std::vector<double>)>& f) {
  int dimension = static_cast<int>(limits.size());
  size_t count = 1;
  std::vector<double> h(dimension);
  for (int i = 0; i < dimension; i++) {
    h[i] = (limits[i].second - limits[i].first) / n;
    count = count * n;
  }

  double result = 0.0;
  std::vector<double> combinations(dimension);
  for (size_t j = 0; j < count; j++) {
    for (int i = 0; i < dimension; i++) {
      combinations[i] = limits[i].first + (j % n) * h[i] + h[i] * 0.5;
    }
    result += f(combinations);
  }

  for (int i = 0; i < dimension; i++) {
    result *= h[i];
  }

  return result;
}

double getParallelIntegrals(
    const int n, const std::vector<std::pair<double, double>>& limits,
    const std::function<double(std::vector<double>)>& f) {
  int size, rank;
  MPI_Comm_size(MPI_COMM_WORLD, &size);
  MPI_Comm_rank(MPI_COMM_WORLD, &rank);

  int dimension = static_cast<int>(limits.size());
  std::vector<double> h(n);
  int count = 1;

  for (int i = 0; i < dimension; i++) {
    h[i] = (limits[i].second - limits[i].first) / n;
    count *= n;
  }

  int delta = count / size;
  int rem = count % size;

  int interval_start = 0;
  if (rank != 0) interval_start = rank * delta + rem;
  int interval_end = (rank + 1) * delta + rem;

  std::vector<double> combinations(dimension);
  double local_result = 0.0;
  for (int j = interval_start; j < interval_end; j++) {
    for (int i = 0; i < dimension; i++) {
      combinations[i] = limits[i].first + (j % n) * h[i] + h[i] * 0.5;
    }
    local_result += f(combinations);
  }

  double global_result = 0.0;
  MPI_Reduce(&local_result, &global_result, 1, MPI_DOUBLE, MPI_SUM, 0,
             MPI_COMM_WORLD);

  if (rank == 0) {
    for (int i = 0; i < dimension; i++) {
      global_result *= h[i];
    }
  }

  return global_result;
}

\end{lstlisting}
main.cpp
\begin{lstlisting}
// Copyright 2021 Gordey Maria
#include <gtest/gtest.h>

#include <gtest-mpi-listener.hpp>

#include "./riemann_sum.h"

TEST(RIEMANN_SUM_MPI, TEST_Function_1) {
  int rank;
  MPI_Comm_rank(MPI_COMM_WORLD, &rank);

  double start, end, stime;

  std::vector<std::pair<double, double>> limits(2);
  limits = {{0, 2}, {0, 1}};
  int n = 200;
  double reference_result;

  const std::function<double(std::vector<double>)> f =
      [](std::vector<double> vec) {
        double x = vec[0];
        double y = vec[1];
        return x + y * y;
      };

  if (rank == 0) {
    start = MPI_Wtime();
    reference_result = getSequentialIntegrals(n, limits, f);
    end = MPI_Wtime();
    stime = end - start;
    std::cout << "Seqeuntial time: " << stime << std::endl;
    std::cout << "Sequential result: " << reference_result << std::fixed
              << std::endl;
    start = MPI_Wtime();
  }

  double parallel_result = getParallelIntegrals(n, limits, f);

  if (rank == 0) {
    end = MPI_Wtime();
    double ptime = end - start;
    std::cout << "Parallel time: " << ptime << std::endl;
    std::cout << "Parallel result: " << parallel_result << std::fixed
              << std::endl;
    std::cout << "Speedup is " << stime / ptime << std::endl;

    ASSERT_NEAR(reference_result, parallel_result, eps);
  }
}

TEST(RIEMANN_SUM_MPI, TEST_Function_2) {
  int rank;
  MPI_Comm_rank(MPI_COMM_WORLD, &rank);

  double start, end, stime;

  std::vector<std::pair<double, double>> limits(2);
  limits = {{0, 1}, {0, 1.5}};
  int n = 200;
  double reference_result;

  const std::function<double(std::vector<double>)> f =
      [](std::vector<double> vec) {
        double x = vec[0];
        double y = vec[1];
        return 4 - x * x - y * y;
      };

  if (rank == 0) {
    start = MPI_Wtime();
    reference_result = getSequentialIntegrals(n, limits, f);
    end = MPI_Wtime();
    stime = end - start;
    std::cout << "Seqeuntial time: " << stime << std::endl;
    std::cout << "Sequential result: " << reference_result << std::fixed
              << std::endl;
    start = MPI_Wtime();
  }

  double parallel_result = getParallelIntegrals(n, limits, f);

  if (rank == 0) {
    end = MPI_Wtime();
    double ptime = end - start;
    std::cout << "Parallel time: " << ptime << std::endl;
    std::cout << "Parallel result: " << parallel_result << std::fixed
              << std::endl;
    std::cout << "Speedup is " << stime / ptime << std::endl;

    ASSERT_NEAR(reference_result, parallel_result, eps);
  }
}

TEST(RIEMANN_SUM_MPI, TEST_Function_3) {
  int rank;
  MPI_Comm_rank(MPI_COMM_WORLD, &rank);

  double start, end, stime;

  std::vector<std::pair<double, double>> limits(3);
  limits = {{-2, 3}, {0, 1}, {0, 2.4}};
  int n = 100;
  double reference_result;

  const std::function<double(std::vector<double>)> f =
      [](std::vector<double> vec) {
        double x = vec[0];
        double y = vec[1];
        double z = vec[2];
        return sin(x) * y * z * z;
      };

  if (rank == 0) {
    start = MPI_Wtime();
    reference_result = getSequentialIntegrals(n, limits, f);
    end = MPI_Wtime();
    stime = end - start;
    std::cout << "Seqeuntial time: " << stime << std::endl;
    std::cout << "Sequential result: " << reference_result << std::fixed
              << std::endl;
    start = MPI_Wtime();
  }

  double parallel_result = getParallelIntegrals(n, limits, f);

  if (rank == 0) {
    end = MPI_Wtime();
    double ptime = end - start;
    std::cout << "Parallel time: " << ptime << std::endl;
    std::cout << "Parallel result: " << parallel_result << std::fixed
              << std::endl;
    std::cout << "Speedup is " << stime / ptime << std::endl;

    ASSERT_NEAR(reference_result, parallel_result, eps);
  }
}

TEST(RIEMANN_SUM_MPI, TEST_Function_4) {
  int rank;
  MPI_Comm_rank(MPI_COMM_WORLD, &rank);

  double start, end, stime;

  std::vector<std::pair<double, double>> limits(3);
  limits = {{0, 1}, {-3, 2.5}, {0, exp(2)}};
  int n = 100;
  double reference_result;

  const std::function<double(std::vector<double>)> f =
      [](std::vector<double> vec) {
        double x = vec[0];
        double y = vec[1];
        double z = vec[2];
        return x * x * y * cos(z);
      };

  if (rank == 0) {
    start = MPI_Wtime();
    reference_result = getSequentialIntegrals(n, limits, f);
    end = MPI_Wtime();
    stime = end - start;
    std::cout << "Seqeuntial time: " << stime << std::endl;
    std::cout << "Sequential result: " << reference_result << std::fixed
              << std::endl;
    start = MPI_Wtime();
  }

  double parallel_result = getParallelIntegrals(n, limits, f);

  if (rank == 0) {
    end = MPI_Wtime();
    double ptime = end - start;
    std::cout << "Parallel time: " << ptime << std::endl;
    std::cout << "Parallel result: " << parallel_result << std::fixed
              << std::endl;
    std::cout << "Speedup is " << stime / ptime << std::endl;

    ASSERT_NEAR(reference_result, parallel_result, eps);
  }
}

TEST(RIEMANN_SUM_MPI, TEST_Function_5) {
  int rank;
  MPI_Comm_rank(MPI_COMM_WORLD, &rank);

  double start, end, stime;

  std::vector<std::pair<double, double>> limits(3);
  limits = {{0, 2}, {-1, 2}, {1, exp(1)}};
  int n = 100;
  double reference_result;

  const std::function<double(std::vector<double>)> f =
      [](std::vector<double> vec) {
        double x = vec[0];
        double y = vec[1];
        double z = vec[2];
        return x * y * y * y / z;
      };

  if (rank == 0) {
    start = MPI_Wtime();
    reference_result = getSequentialIntegrals(n, limits, f);
    end = MPI_Wtime();
    stime = end - start;
    std::cout << "Seqeuntial time: " << stime << std::endl;
    std::cout << "Sequential result: " << reference_result << std::fixed
              << std::endl;
    start = MPI_Wtime();
  }

  double parallel_result = getParallelIntegrals(n, limits, f);

  if (rank == 0) {
    end = MPI_Wtime();
    double ptime = end - start;
    std::cout << "Parallel time: " << ptime << std::endl;
    std::cout << "Parallel result: " << parallel_result << std::fixed
              << std::endl;
    std::cout << "Speedup is " << stime / ptime << std::endl;

    ASSERT_NEAR(reference_result, parallel_result, eps);
  }
}

int main(int argc, char** argv) {
  ::testing::InitGoogleTest(&argc, argv);
  MPI_Init(&argc, &argv);

  ::testing::AddGlobalTestEnvironment(new GTestMPIListener::MPIEnvironment);
  ::testing::TestEventListeners& listeners =
      ::testing::UnitTest::GetInstance()->listeners();

  listeners.Release(listeners.default_result_printer());
  listeners.Release(listeners.default_xml_generator());

  listeners.Append(new GTestMPIListener::MPIMinimalistPrinter);
  return RUN_ALL_TESTS();
}

\end{lstlisting}

\end{document}