\documentclass{report}

\usepackage[warn]{mathtext}
\usepackage[T2A]{fontenc}
\usepackage[utf8]{luainputenc}
\usepackage[english, russian]{babel}
\usepackage[pdfpagemode=UseNone,colorlinks,allcolors=black]{hyperref}
\usepackage[12pt]{extsizes}
\usepackage{listings}
\usepackage{color}
\usepackage{geometry}
\usepackage{enumitem}
\usepackage{multirow}
\usepackage{graphicx}
\usepackage{indentfirst}
\usepackage{amsmath}

\geometry{a4paper,top=2cm,bottom=2cm,left=2.5cm,right=1.5cm}
\setlength{\parskip}{0.5cm}
\setlist{nolistsep, itemsep=0.3cm,parsep=0pt}

\usepackage{listings}
\lstset{language=C++,
        basicstyle=\footnotesize,
		keywordstyle=\color{blue}\ttfamily,
		stringstyle=\color{red}\ttfamily,
		commentstyle=\color{green}\ttfamily,
		morecomment=[l][\color{red}]{\#}, 
		tabsize=4,
		breaklines=true,
  		breakatwhitespace=true,
  		title=\lstname,       
}

\makeatletter
\renewcommand\@biblabel[1]{#1.\hfil}
\makeatother

\begin{document}

\begin{titlepage}

\begin{center}
Министерство науки и высшего образования Российской Федерации
\end{center}

\begin{center}
Федеральное государственное автономное образовательное учреждение высшего образования \\
Национальный исследовательский Нижегородский государственный университет им. Н.И. Лобачевского
\end{center}

\begin{center}
Институт информационных технологий, математики и механики
\end{center}

\vspace{4em}

\begin{center}
\textbf{\LargeОтчет по лабораторной работе} \\
\end{center}
\begin{center}
\textbf{\Large«Маркировка компонент на бинарном изображении (черные области соответствуют объектам, белые – фону)»} \\
\end{center}

\vspace{4em}

\newbox{\lbox}
\savebox{\lbox}{\hbox{text}}
\newlength{\maxl}
\setlength{\maxl}{\wd\lbox}
\hfill\parbox{7cm}{
\hspace*{5cm}\hspace*{-5cm}\textbf{Выполнил:} \\ студентка группы 381906-2 \\ Воробьева А.И.\\
\\
\hspace*{5cm}\hspace*{-5cm}\textbf{Проверил:}\\ доцент кафедры МОСТ, \\ кандидат технических наук \\ Сысоев А. В.\\
}
\vspace{\fill}

\begin{center} Нижний Новгород \\ 2021 \end{center}

\end{titlepage}

\setcounter{page}{2}

% Содержание
\tableofcontents
\newpage

% Введение
\section*{Введение}
\addcontentsline{toc}{section}{Введение}
\par Под выделением связных компонент понимают присвоение
уникальной метки каждому объекту изображения. При
последующем анализе данные метки служат в качестве
идентификаторов при обращении к объектам. Это делает
операцию выделения связных компонент неотъемлемой
частью почти всех приложений распознавания образов и
компьютерного зрения. Например, перед тем как компьютер
может определить или классифицировать любой объект
изображения (автомобиль, человека, внутренний орган)
группы смежных пикселей должны быть идентифицированы
и промаркированы. Каждая выделенная группа пикселей
соответствует объекту на изображении. Такая группировка
смежных пикселей позволяет исследователю получить
необходимые для последующего анализа свойства объектов,
такие как высота, ширина, периметр, площадь. Очевидно, что
задача выделения связных компонент является
фундаментальной задачей обработки изображений. Следует
отметить, что для многих приложений данная операция
является наиболее времязатратной, т. е. критической. По этим
причинам выделение связных компонент до сих пор остается
активной областью исследований. В
настоящей работе проводится классификация и
сравнительный анализ алгоритмов выделения связных
компонент с целью определения алгоритма, имеющего
наилучшую производительность при работе с
изображениями, объекты которых представлены
совокупностью линий, имеющих относительно одинаковую
толщину на всем протяжении. 

\newpage

% Постановка задачи
\section*{Постановка задачи}
\addcontentsline{toc}{section}{Постановка задачи}
\par В данной лабораторной работе требуется реализовать последовательный и параллельные алгоритмы маркировки компонент на бинарном изображении.
\par Для оценки эффективности работы программы нужно произвести серию экспериментов, сравнивающих время выполнения последовательной и параллельной версии алгоритма, сравнить полученные результаты и сделать выводы.
\par Параллельный алгоритм должен быть реализован при помощи технологии MPI.
\newpage

% Описание алгоритма
\section*{Описание алгоритма}
\addcontentsline{toc}{section}{Описание алгоритма}
\par Для маркировки будем использовать, так
называемый, двухпроходный алгоритм маркировки. Данный алгоритм использует систему разделённых множеств. Основная идея заключается в том, чтобы пройтись по всему изображению целиком лишь 2 раза. Поэтому данный алгоритм можно разбить на 2 части. Первую часть назовём "первый проход", а вторую часть алгоритма -
"второй проход". Рассмотрим их по отдельности:
\par Обозначим: A - метка левого пикселя в карте временных меток,
B - метка верхнего пикселя в карте временных меток.
\par • "Первый проход".
\begin{itemize}
\item Так как алгоритм работает с СРМ, необходимо создать N синглетонов,
по одному на каждый пиксель.
\item Проходим в цикле по всем пикселям, для которых I(x, y) не равно 0, и
рассматриваем значения пикселей с позициями (x - 1, y) и (x, y - 1), т.е.
левый и верхний пиксели соответственно (так как мы проходимся слева
направо и сверху вниз, именно эти пиксели мы точно уже прошли и о
них нам уже что-то может быть известно). Дальше может быть один из
следующих вариантов:
\end{itemize}
\begin{itemize}
\item Если A равно 0 и B равно 0, то текущий пиксель принадлежит не помеченной компоненте, поэтому мы помечаем данный пиксель в карте
временных меток значением равным индексу и инкрементируем число найденных компонент.
\item Если A равно 0, а B не равно 0, то считаем, что текущий пиксель при5
надлежит уже помеченной компоненте, в которую входит пиксель B
(верхний пиксель), поэтому в карте временных меток текущему пикселю также ставим значение B, соответствующее значению в карте
временных меток верхнего пикселя.
\item Если A не равно 0, а B равно 0, то считаем, что текущий пиксель принадлежит уже помеченной компоненте, в которую входит пиксель A
(левый пиксель), поэтому в карте временных меток текущему пикселю также ставим значение A, соответствующее значению в карте
временных меток левого пикселя.
\item Если A не равно 0, B не равно 0 и A равно B, то считаем, что текущий
пиксель принадлежит уже помеченной компоненте, в которую входят
и пиксель A (левый пиксель), и пиксель B (верхний пиксель), поэтому
в карте временных меток текущему пикселю также ставим значение
A (можно поставить B, потому что они равны).
\item Если A не равно 0, B не равно 0 и A не равно B, то считаем, что
текущий пиксель является "связующим" для двух частей одной и той
же компоненты, одной части которой принадлежит левый пиксель
A, а другой - верхний пиксель B. Поэтому необходимо первым делом
узнать не принадлежат ли A и B одному и тому же множеству в
СРМ, соответственно воспользоваться методом Find() в СРМ для A
и для B.
Если они принадлежат одному множеству, то мы ставим для данного
пикселя в карте временных меток метку, соответствующую названию
множества. Фактически A и B принадлежат одной и той же компоненте, но формально в карте временных меток у них разные значения. Это и есть причина, по которой данный массив называется
картой временных меток, а не просто картой меток.
\end{itemize}
\par Если они принадлежат разным множествам, то нужно их объединить
в одно, воспользовавшись методом Unite() в СРМ для множеств, которым принадлежат A и B соответственно. А затем мы ставим для
данного пикселя в карте временных меток метку, соответствующую
названию объединённого множества и декрементируем число компонент.
\par – В конце алгоритма мы будем точно знать число компонент на изображении. Также у нас будет карта временных меток и СРМ, по которым
"второй проход" будет собирать финальную версию меток.
\par • "Второй проход".
\par – Проходим в цикле по всем меткам в карте временных меток и записываем в результирующее изображение в соответствующую позицию на6
звание множества в СРМ, которому принадлежит данная метка (оно
уникально и единственно для каждого множества и, как правило, его
берут равным значению минимального элемента множества), т.е. воспользуемся методом Name() в СРМ для текущей метки.
\par – После прохода по циклу у нас будет финальное промаркированное изображение, которое и будет итоговым результатом работы алгоритма.

\newpage

% Схема распараллеливания
\section*{Схема распараллеливания}
\addcontentsline{toc}{section}{Схема распараллеливания}
\par Так как всё изображение хранится в процессе с рангом 0, первым шагом
алгоритма является разбиение изображения (на группы из нескольких строк)
и отправка его частей другим процессам.
\par Вторым шагом мы на каждом процессе запускаем алгоритм "первого прохода" и получаем карту временных меток размерами, соответствующими полученной части изображения, СРМ и число компонент для данного участка изображения. Именно на этом шаге и будет происходить ускорение из-за того, что
нам необходимо будет при формировании СРМ проходиться по много раз по
полученному процессом части изображения. Связано это с тем, что в данной
реализации СРМ операции поиска и объединения в общем случае зависят от
размеров изображения. И при наличии большого количества таких операций,
мы получаем ускорение по сравнению с последовательным алгоритмом.
\par Третьим шагом мы объединяем локальные СРМ и локальные карты временных меток в единые для всего изображения глобальные СРМ и глобальную
карту временных меток. Однако из-за того, что мы разбили исходное изображение на части, мы могли потерять в СРМ связь между частями компонент,
по которым прошёлся "разрез". Для такой компоненты часть оказалась в одном процессе, а другая часть - в другом. Эти участки также можно назвать
"склейками". Их число будет на единицу меньше числа процессов. Для данных
"склеек"требуются дополнительные вычисления для выявления связей между
частями компонент (для этого достаточно рассмотреть лишь крайние строки
переданных процессам частей) и добавлением, в случае необходимости, в СРМ.
\par Финальным шагом является вызов "второго прохода", тем самым получив
итоговую маркировку изображения. К сожалению, данную часть алгоритма
нельзя распараллелить. В ней реализуется одноразовый проход по карте временных меток с установкой финальных меток, сложность которой равняется
O(1), поэтому, вообще говоря, "второй проход" - это линейная задача, а у подобных задач невозможно добиться ускорения за счёт распараллеливания.
\newpage

% Описание программной реализации
\section*{Описание программной реализации}
\addcontentsline{toc}{section}{Описание программной реализации}
Программа состоит из заголовочного файла component\_labeling.h и двух файлов исходного кода component\_labeling.cpp и main.cpp.
\par В заголовочном файле находятся прототипы функций для последовательного и параллельного вычисления многомерных интегралов.
\par Функция для последовательного алгоритма:
\begin{lstlisting}
pair<vector<int>, int> ComponentLabeling(vector<int> image, int width,
                                         int height);
\end{lstlisting}
Входными параметрами функции является изображение, ширина и высота.
\par Функция для параллельного
алгоритма:
\begin{lstlisting}
pair<vector<int>, int> ComponentLabelingParallel(vector<int> image, int width,
                                                 int height);
\end{lstlisting}
Входные параметры данной функции совпадают с входными параметрами функции для последовательного алгоритма.
\par В файле исходного кода component\_labeling.cpp содержится реализация функций, объявленных в заголовочном файле. В файле исходного кода main.cpp содержатся тесты для проверки корректности программы.
\newpage

% Подтверждение корректности
\section*{Подтверждение корректности}
\addcontentsline{toc}{section}{Подтверждение корректности}
Для подтверждения корректности работы данной программы с помощью фрэймфорка Google Test была разработана серия тестов. В каждом из тестов вычисляется значение контрольного  интеграла заданной размерности при помощи последовательного и параллельного алгоритмов, подсчитывается время работы обоих алгоритмов, находится ускорение делением времени работы последовательного алгоритма на время работы параллельного. Результаты вычисления интеграла последовательным и параллельным способом сравниваются между собой в пределах погрешности, после чего можно сделать выводы об эффективности программы.

\par Успешное прохождение разработанных мной тестов подтверждает корректность работы программы.
\newpage

% Результаты экспериментов
\section*{Результаты экспериментов}
\addcontentsline{toc}{section}{Результаты экспериментов}
Вычислительные эксперименты для оценки эффективности работы параллельного алгоритма проводились на ПК со следующими характеристиками:
\begin{itemize}
\item Процессор: Intel Core i3-4170, 3.5 ГГц, количество ядер: 4;
\item Оперативная память: 8 ГБ DDR3, 1600  МГц;
\item Операционная система: Windows 10 Home.
\end{itemize}

\par Результаты экспериментов представлены в Таблице 1.

\begin{table}[!h]
\caption{Результаты вычислительных экспериментов в тесте 4}
\centering
\begin{tabular}{| p{2cm} | p{3cm} | p{4cm} | p{2cm} |}
\hline
Количество процессов & Время работы последовательного алгоритма (в секундах) & Время работы параллельного алгоритма (в секундах) & Ускорение  \\[5pt]
\hline
1        & 0.8591        & 0.892684     & 0.962379       \\
2        & 0.902287        & 0.71415     & 1.26344       \\
3        & 0.915743        & 0.676334     & 1.35398       \\
4        & 0.900046        & 0.659483     & 1.36477       \\
8        & 0.843329        & 0.804964     & 1.04766	  \\
\hline
\end{tabular}
\end{table}
По данным, полученным в результате экспериментов, можно сделать вывод о том,
что параллельный алгоритм работает быстрее, чем последовательный в 1.35 раз.

\newpage

% Заключение
\section*{Заключение}
\addcontentsline{toc}{section}{Заключение}
В ходе выполнения данной лабораторной работы были разработаны последовательный и параллельный алгоритмы маркировки компонент на бинарном изображении. После выполенения серии экспериментов можно сделать вывод о том, что параллельный алгоритм работает быстрее последовательного, что доказывает эффективность параллельного алгоритма по сравнению с последовательным.
\par Были разработаны и доведены до успешного выполнения тесты,  с использованием Google C++ Testing Framework, которые  подтвердиди корректность выполнения  программы.
\newpage

% Литература
\section*{Литература}
\addcontentsline{toc}{section}{Литература}
\begin{enumerate}
\item Гергель В. П. Теория и практика параллельных вычислений. – 2007.
\itemГергель В. П., Стронгин Р. Г. Основы параллельных вычислений для многопроцессорных вычислительных систем. – 2003.

\end{enumerate} 
\newpage

% Приложение
\section*{Приложение}
\addcontentsline{toc}{section}{Приложение}

component\_labeling.cpp
\begin{lstlisting}
// Copyright 2021 Vorobyova Anna
#include "../../modules/task_3/vorobyova_a_component_labeling/component_labeling.h"

#include <mpi.h>

std::vector<int> getRandomImage(int n, int m) {
  std::mt19937 gen(time(0));
  std::uniform_int_distribution<> uniform(0, 1);
  vector<int> image(n * m);
  for (int i = 0; i < n * m; i++) {
    image[i] = uniform(gen);
  }
  return image;
}

pair<vector<int>, pair<vector<int>, int> > partOne(vector<int> image, int n,
                                                   int m, int start) {
  int c = 0;
  int matrix_size = n * m;
  vector<int> srm(matrix_size);
  vector<int> map(image.begin(), image.begin() + matrix_size);
  for (int x = 0; x < matrix_size; x++) srm[x] = x + start;

  for (int i = 0; i < m; i++) {
    for (int j = 0; j < n; j++) {
      int index = i * n + j;
      if (map[index] != 0) {
        int left, upper;
        if (index < n) {
          left = 0;
        } else {
          left = map[index - n];
        }
        if (((index < 1) || ((index - 1) / n != i))) {
          upper = 0;
        } else {
          upper = map[index - 1];
        }

        if ((left == 0) && (upper == 0)) {
          map[index] = index + start + 1;
          c++;
        }
        if ((left == 0) && (upper != 0)) map[index] = upper;
        if ((left != 0) && (upper == 0)) map[index] = left;
        if ((left != 0) && (upper != 0)) {
          if (left == upper) {
            map[index] = left;
          } else {
            int max_pixel = max(left, upper);
            while (srm[max_pixel - start] != max_pixel)
              max_pixel = srm[max_pixel - start];

            int min_pixel = min(left, upper);
            while (srm[min_pixel - start] != min_pixel)
              min_pixel = srm[min_pixel - start];

            if (max_pixel != min_pixel) {
              srm[max_pixel - start] = min_pixel;
              c--;
            }

            map[index] = min_pixel;
          }
        }
      }
    }
  }

  return make_pair(map, make_pair(srm, c));
}

vector<int> partTwo(vector<int> map, vector<int> srm, int n, int m) {
  int matrix_size = n * m;
  vector<int> final_matrix(matrix_size);
  int total = matrix_size;
  for (int pos = 0; pos < total - 1; pos++) {
    int pixel = map[pos];
    if (pixel != 0) {
      if (srm[pixel] == pixel) {
        final_matrix[pos] = pixel;
      } else {
        while (srm[pixel] != pixel) pixel = srm[pixel];
        final_matrix[pos] = pixel;
      }
    }
  }
  return final_matrix;
}

pair<vector<int>, int> ComponentLabeling(vector<int> image, int n, int m) {
  pair<vector<int>, pair<vector<int>, int> > part_one = partOne(image, n, m);

  vector<int> map = part_one.first;
  vector<int> srm = part_one.second.first;
  int c = part_one.second.second;

  vector<int> final_matrix = partTwo(map, srm, n, m);

  return make_pair(final_matrix, c);
}

pair<vector<int>, int> ComponentLabelingParallel(vector<int> image, int n,
                                                 int m) {
  int size, rank;
  MPI_Comm_size(MPI_COMM_WORLD, &size);
  MPI_Comm_rank(MPI_COMM_WORLD, &rank);

  if (size == 1) {
    return ComponentLabeling(image, n, m);
  }

  int matrix_size = n * m;
  const int delta = m / size * n;
  const int remainder = (m % size) * n;

  if (delta == 0) {
    if (rank == 0) {
      return ComponentLabeling(image, n, m);
    } else {
      return make_pair(vector<int>{}, 0);
    }
  }

  if (rank == 0) {
    for (int proc = 1; proc < size; proc++)
      MPI_Send(image.data() + remainder + proc * delta, delta, MPI_INT, proc, 0,
               MPI_COMM_WORLD);
  }

  vector<int> local_matrix(delta + remainder);

  if (rank == 0) {
    local_matrix =
        vector<int>(image.cbegin(), image.cbegin() + delta + remainder);
  } else {
    MPI_Recv(local_matrix.data(), delta, MPI_INT, 0, 0, MPI_COMM_WORLD,
             MPI_STATUSES_IGNORE);
  }

  int local_height = 0, local_label = delta * rank;
  if (rank == 0) {
    local_height = (remainder + delta) / n;
  } else {
    local_height = delta / n;
    local_label += remainder;
  }
  pair<vector<int>, pair<vector<int>, int> > part_one =
      partOne(local_matrix, n, local_height, local_label);

  vector<int> map = part_one.first, srm = part_one.second.first;
  int local_labels_counts = part_one.second.second;

  vector<int> starts(size), deltas(size);
  deltas[0] = remainder + delta;
  deltas[1] = delta;
  starts[1] = remainder + delta;
  for (int proc = 2; proc < size; proc++) {
    starts[proc] = starts[proc - 1] + delta;
    deltas[proc] = delta;
  }

  vector<int> global_srm(matrix_size);
  int temp = delta;
  if (rank == 0) temp += remainder;

  MPI_Gatherv(srm.data(), temp, MPI_INT, global_srm.data(), deltas.data(),
              starts.data(), MPI_INT, 0, MPI_COMM_WORLD);

  vector<int> global_map(matrix_size);
  MPI_Gatherv(map.data(), temp, MPI_INT, global_map.data(), deltas.data(),
              starts.data(), MPI_INT, 0, MPI_COMM_WORLD);

  int global_c = 0;
  MPI_Reduce(&local_labels_counts, &global_c, 1, MPI_INT, MPI_SUM, 0,
             MPI_COMM_WORLD);

  vector<int> global_matrix(matrix_size);
  if (rank == 0) {
    for (int j = 1; j < size; j++) {
      int row_two = delta * j;
      if (j != 0) {
        row_two += remainder;
      }
      int row_one = row_two - n;

      for (int x = 0; x < n; x++) {
        int left = global_map[row_one + x];
        int upper = global_map[row_two + x];
        if ((left != 0) && (upper != 0)) {
          int left_set = global_srm[left];
          int upper_set = global_srm[upper];
          if (left_set != upper_set) {
            int max_pixel = max(left, upper);

            while (global_srm[max_pixel] != max_pixel)
              max_pixel = global_srm[max_pixel];

            int min_pixel = min(left, upper);

            while (global_srm[min_pixel] != min_pixel)
              min_pixel = global_srm[min_pixel];

            if (max_pixel != min_pixel) {
              global_srm[max_pixel] = min_pixel;
              global_c--;
            }
          }
        }
      }
    }
    global_matrix = partTwo(global_map, global_srm, n, m);
  }

  return make_pair(global_matrix, global_c);
}
\end{lstlisting}
main.cpp
\begin{lstlisting}
// Copyright 2021 Vorobyova Anna
#include <gtest/gtest.h>

#include <gtest-mpi-listener.hpp>

#include "./component_labeling.h"

TEST(COMPONENT_LABELING_PARALLEL, TEST_1) {
  int rank;
  MPI_Comm_rank(MPI_COMM_WORLD, &rank);

  vector<int> image;
  int n = 200, m = 300;

  if (rank == 0) {
    image = getRandomImage(n, m);
  }

  pair<vector<int>, int> parallel_result =
      ComponentLabelingParallel(image, n, m);

  if (rank == 0) {
    pair<vector<int>, int> reference_result = ComponentLabeling(image, n, m);
    ASSERT_EQ(parallel_result.second, reference_result.second);
  }
}

TEST(COMPONENT_LABELING_PARALLEL, TEST_2) {
  int rank;
  MPI_Comm_rank(MPI_COMM_WORLD, &rank);

  vector<int> image;
  int n = 500, m = 600;

  if (rank == 0) {
    image = getRandomImage(n, m);
  }

  pair<vector<int>, int> parallel_result =
      ComponentLabelingParallel(image, n, m);

  if (rank == 0) {
    pair<vector<int>, int> reference_result = ComponentLabeling(image, n, m);
    ASSERT_EQ(parallel_result.second, reference_result.second);
  }
}

TEST(COMPONENT_LABELING_PARALLEL, TEST_3) {
  int rank;
  MPI_Comm_rank(MPI_COMM_WORLD, &rank);

  vector<int> image;
  int n = 400, m = 170;

  if (rank == 0) {
    image = getRandomImage(n, m);
  }

  pair<vector<int>, int> parallel_result =
      ComponentLabelingParallel(image, n, m);

  if (rank == 0) {
    pair<vector<int>, int> reference_result = ComponentLabeling(image, n, m);
    ASSERT_EQ(parallel_result.second, reference_result.second);
  }
}

TEST(COMPONENT_LABELING_PARALLEL, TEST_4) {
  int rank;
  MPI_Comm_rank(MPI_COMM_WORLD, &rank);

  vector<int> image;
  int n = 239, m = 547;

  if (rank == 0) {
    image = getRandomImage(n, m);
  }

  pair<vector<int>, int> parallel_result =
      ComponentLabelingParallel(image, n, m);

  if (rank == 0) {
    pair<vector<int>, int> reference_result = ComponentLabeling(image, n, m);
    ASSERT_EQ(parallel_result.second, reference_result.second);
  }
}

TEST(COMPONENT_LABELING_PARALLEL, TEST_5) {
  int rank;
  MPI_Comm_rank(MPI_COMM_WORLD, &rank);

  vector<int> image;
  int n = 1300, m = 600;

  if (rank == 0) {
    image = getRandomImage(n, m);
  }

  double start = MPI_Wtime();
  pair<vector<int>, int> parallel_result =
      ComponentLabelingParallel(image, n, m);
  double end = MPI_Wtime();

  if (rank == 0) {
    double ptime = end - start;

    start = MPI_Wtime();
    pair<vector<int>, int> reference_result = ComponentLabeling(image, n, m);
    end = MPI_Wtime();
    double stime = end - start;

    std::cout << stime / ptime << std::endl;

    ASSERT_EQ(parallel_result.second, reference_result.second);
  }
}

int main(int argc, char** argv) {
  ::testing::InitGoogleTest(&argc, argv);
  MPI_Init(&argc, &argv);

  ::testing::AddGlobalTestEnvironment(new GTestMPIListener::MPIEnvironment);
  ::testing::TestEventListeners& listeners =
      ::testing::UnitTest::GetInstance()->listeners();

  listeners.Release(listeners.default_result_printer());
  listeners.Release(listeners.default_xml_generator());

  listeners.Append(new GTestMPIListener::MPIMinimalistPrinter);
  return RUN_ALL_TESTS();
}
\end{lstlisting}
component\_labeling.h
\begin{lstlisting}
// Copyright 2021 Vorobyova Anna
#ifndef MODULES_TASK_3_VOROBYOVA_A_COMPONENT_LABELING_COMPONENT_LABELING_H_
#define MODULES_TASK_3_VOROBYOVA_A_COMPONENT_LABELING_COMPONENT_LABELING_H_

#include <algorithm>
#include <ctime>
#include <iostream>
#include <random>
#include <utility>
#include <vector>

using std::max;
using std::min;
using std::pair;
using std::vector;

vector<int> getRandomImage(int width, int height);
pair<vector<int>, pair<vector<int>, int> > partOne(vector<int> image, int width,
                                                   int height,
                                                   int first_label = 0);
vector<int> partTwo(vector<int> map, vector<int> disjoint_sets, int width,
                    int height);
pair<vector<int>, int> ComponentLabeling(vector<int> image, int width,
                                         int height);
pair<vector<int>, int> ComponentLabelingParallel(vector<int> image, int width,
                                                 int height);

#endif  // MODULES_TASK_3_VOROBYOVA_A_COMPONENT_LABELING_COMPONENT_LABELING_H_
\end{lstlisting}
\end{document}