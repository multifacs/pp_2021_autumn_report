\documentclass{report}

\usepackage[warn]{mathtext}
\usepackage[T2A]{fontenc}
\usepackage[utf8]{luainputenc}
\usepackage[english, russian]{babel}
\usepackage[pdftex]{hyperref}
\usepackage[12pt]{extsizes}
\usepackage{listings}
\usepackage{color}
\usepackage{geometry}
\usepackage{enumitem}
\usepackage{multirow}
\usepackage{graphicx}
\usepackage{indentfirst}
\usepackage{amsmath}

\geometry{a4paper,top=2cm,bottom=2cm,left=2.5cm,right=1.5cm}
\setlength{\parskip}{0.5cm}
\setlist{nolistsep, itemsep=0.3cm,parsep=0pt}

\usepackage{listings}
\lstset{language=C++,
        basicstyle=\footnotesize,
		keywordstyle=\color{blue}\ttfamily,
		stringstyle=\color{red}\ttfamily,
		commentstyle=\color{green}\ttfamily,
		morecomment=[l][\color{red}]{\#}, 
		tabsize=4,
		breaklines=true,
  		breakatwhitespace=true,
  		title=\lstname,       
}

\makeatletter
\renewcommand\@biblabel[1]{#1.\hfil}
\makeatother

\begin{document}

\begin{titlepage}

\begin{center}
Министерство науки и высшего образования Российской Федерации
\end{center}

\begin{center}
Федеральное государственное автономное образовательное учреждение высшего образования \\
Национальный исследовательский Нижегородский государственный университет им. Н.И. Лобачевского
\end{center}

\begin{center}
Институт информационных технологий, математики и механики
\end{center}

\vspace{4em}

\begin{center}
\textbf{\LargeОтчет по лабораторной работе} \\
\end{center}
\begin{center}
\textbf{\Large«Вычисление многомерных интегралов с использованием многошаговой схемы 
(метод трапеций)»} \\
\end{center}

\vspace{4em}

\newbox{\lbox}
\savebox{\lbox}{\hbox{text}}
\newlength{\maxl}
\setlength{\maxl}{\wd\lbox}
\hfill\parbox{7cm}{
\hspace*{5cm}\hspace*{-5cm}\textbf{Выполнилa:} \\ студентка группы 381906-1 \\ Вотякова Д.С. \\
\\
\hspace*{5cm}\hspace*{-5cm}\textbf{Проверил:}\\ доцент кафедры МОСТ, \\ кандидат технических наук \\ Сысоев А. В.\\
}
\vspace{\fill}

\begin{center} Нижний Новгород \\ 2021 \end{center}

\end{titlepage}

\setcounter{page}{2}

% Содержание
\tableofcontents
\newpage

% Введение
\section*{Введение}
\addcontentsline{toc}{section}{Введение}
\par В данной лабораторной работе мне было предложено решить одну из известных задач математики, задача–вычисление значений определенных интегралов.
Задача вычисления определенного интеграла I для некоторой заданной на отрезке [a, b] функции f(x) является классической задачей математического анализа. Известно,что если функция f(x) ограничена на [a, b] и имеет на нем конечное число точек разрыва, то она интегрируема на этом отрезке.Такое значение дейсвительно существует, единственно и может быть формально получено по определению как\\
\begin{equation}\iiint{f(x_1,\dots,x_k) dx_1 \dots dx_k} = h_i\sum_{i=1}^{N}{f(p_{i})} \end{equation}
В данной лабораторной работе изучаются подходы к распараллеливанию метода трапеций,идеи по алгоритмической оптимизации, приводящие к уменьшению времени вычислений, вопросы параллельной отладки.
\newpage

% Постановка задачи
\section*{Постановка задачи}
\addcontentsline{toc}{section}{Постановка задачи}
\par В данной лабораторной работе я буду реализовывать последовательный и параллельный алгоритмы для вычисления многомерных  интегралов методом трапеций.

\par Параллельный алгоритм должен быть реализован  при помощи технологии MPI. А для проверки корректности работы алгоритмов будем использовать Google C++ Testing Framework.
\newpage

% Описание алгоритма
\section*{Описание алгоритма}
\addcontentsline{toc}{section}{Описание алгоритма}
\par \addcontentsline{toc}{section}{Метод решения}
Алгоритм вычисления интегралов методом трапеций:
\parПусть функция y = f(x) непрерывна на отрезке [a; b]. Нам требуется вычислить определенный интеграл.

\par1) Весь участок [a, b] делим на n равных интервалов с шагом h = (b - a)/ n.
\par2) Вычисляем значение подынтегральной функции в каждой узловой точке.
\par3) На каждом шаге подынтегральную функцию f(x) аппроксимируем прямой, соединяющей две соседние узловые точки. В результате вся подынтегральная функция на участке [a,b] заменяется ломаной линией проходящей через все узловые точки.
\par4) Вычисляем площадь каждой частичной трапеции.
\par5) Приближенное значение интеграла равно сумме площадей частичных трапеций.

\newpage

% Схема распараллеливания
\section*{Схема распараллеливания}
\addcontentsline{toc}{section}{Схема распараллеливания}
\par
 Мы заводим локальную переменную для вычисления площади трапеции,так называемого сегмента в коде программы.Заданное количество сегментов будут распределяться между определенным количеством процессов, каждая локальная переменная будет считаться в каждом процессе,т.е каждый процесс будет считать только свои сегменты,после чего с помощью специальных функций MPI все локальные перемнные собираются в процессе с нулевым рангом,т.е в главном процессе,где тот в свою очередь складывает локальные переменные в глобальную после чего результат будет умножаться на шаги разбиения сегментов интегрирования и в итоге мы получим приближенное значения интеграла.В этом и состоит основная идея реализации параллельного алгоритма.

\newpage

% Описание программной реализации
\section*{Описание программной реализации}
\addcontentsline{toc}{section}{Описание программной реализации}
Программа состоит из трех файлов trapezoidal.h , trapezoidal.cpp и main.cpp.
\par В файле trapezoidal.h мы храним прототипы функций для последовательного и параллельного алгоритмов.
\par Функция для последовательного алгоритма:
\begin{lstlisting}

 double SequentialTrapezoidal(int segments, ABs ab, func fn);
  
\end{lstlisting}
В качестве входных  параметров мы задаем: число сегментов разбиения, массив пределов  и подынтегральная функция.
\par Функция для параллельного
алгоритма:
\begin{lstlisting}
double ParallelTrapezoidal(int segments, ABs ab, const func fn);
\end{lstlisting}
.В качестве входных  параметров мы задаем: число сегментов разбиения, массив пределов  и подынтегральная функция.
\par В файле  trapezoidal.cpp содержится реализация функций, объявленных в файле trapezoidal.h. В файле  main.cpp мы содержим тесты для проверки корректности работы программы.
\newpage

% Подтверждение корректности
\section*{Подтверждение корректности}
\addcontentsline{toc}{section}{Подтверждение корректности}
 Для проверки корректности работы программы я разработала  тесты с помощью фрэймфорка Google Test. В каждом отдельном тесте мы задаем произвольный интеграл произвольной размерности и каждый тест вычисляет значение данного интеграла при помощи последовательного и параллельного алгоритма,также в каждом тесте мы считаем время работы последовательного и параллельного алгритмов,находим ускорение и затем результаты выполнения двух алгоритмов сравниваются  между собой в точности до эпсилон.

\par Если все тесты проходят успешно,то программа работает корректно.
\newpage

% Результаты экспериментов
\section*{Результаты экспериментов}
\addcontentsline{toc}{section}{Результаты экспериментов}
Эксперименты для оценки эффективности работы параллельного алгоритма проводились на ПК со следующими характеристиками:
\begin{itemize}
\item Процессор:Intel(R) Core(TM) i5-8265U 1.8GHz, количество ядер: 4; логических процессов:8;
\item Оперативная память: 8 ГБ
\item Операционная система: Windows 10 Home.
\end{itemize}

\par Эксперименты проводились на различном числе процессов от 1 до 10 для вычисления двумерного  интеграла  с количеством разбиений, равным 200. 
\par Результаты экспериментов представлены мы можем наблюдать в таблице ниже.

\begin{table}[!h]
\caption{Результаты вычислительных экспериментов в тесте 3}
\centering
\begin{tabular}{| p{2cm} | p{3cm} | p{4cm} | p{2cm} |}
\hline
Количество процессов & Время работы последовательного алгоритма (в секундах) & Время работы параллельного алгоритма (в секундах) & Ускорение  \\[5pt]
\hline
1        & 0.05432        & 0.062841    &1.015634       \\
2        & 0.0834      &0.038933   &  2.042084     \\
3        &  0.0769317      & 0.025528      &3.459823 \\
4        &  0.0984553       &  0.014367     & 4.10728     \\
5        &  0.0862348      & 0.023853   &   3.30754     \\
6        &  0.1837453      &   0.019835   &    8.875354    \\
7        &    0.193467      & 0.019826      &  7.442898     \\
8        &   0.1458342       &   0.019367  &   7.619457   \\
10       &    0.157632       &   0.018432  &   12.331947   \\
\hline
\end{tabular}
\end{table}

\par По данным представленным в таблице можно утверждать,что параллельный алгоритм работает быстрее, чем последовательный. Процессов имеет 8 логических ядер, максимальное ускорение составляет 8 раз. В эксперименте на 10 процессов мы можем заметить,что  результат ускорения превосходит это значение. Это связано с тем,что алгоритмы имеют разную сложность и сложность параллельного алгоритма гораздо меньше последовательного.
\newpage


% Заключение
\section*{Заключение}
\addcontentsline{toc}{section}{Заключение}
В рамках данной лабораторной работы были разработаны последовательный и параллельный алгоритмы вычисления многомерных интегралов методом трапеций. Тесты разработанные с помощью  фрэймфорка Google Test подтвердили корректность работы реализованной программы, а  эксперименты показали, что параллельный алгоритм работает быстрее.
\newpage

% Литература
\section*{Литература}
\addcontentsline{toc}{section}{Литература}
\begin{enumerate}
\item Гергель В. П. Теория и практика параллельных вычислений. – 2007.
\itemГергель В. П., Стронгин Р. Г. Основы параллельных вычислений для многопроцессорных вычислительных систем. – 2003.

\end{enumerate} 
\newpage

% Приложение
\section*{Приложение}
\addcontentsline{toc}{section}{Приложение}

trapezoidal.cpp
\begin{lstlisting}
// Copyright 2021 Votyakova Daria
#include "../../../modules/task_3/votyakova_d_trapezoidal/trapezoidal.h"

#include <mpi.h>

double SequentialTrapezoidal(int segments, ABs ab, func fn) {
  int integral_dim = static_cast<int>(ab.size());
  int it = 1;
  vector<double> height(integral_dim);
  for (int i = 0; i < integral_dim; i++) {
    height[i] = (ab[i].second - ab[i].first) / segments;
    it = it * segments;
  }
  double result = 0.0;
  vector<double> x(integral_dim);
  for (int i = 0; i < it; i++) {
    for (int j = 0; j < integral_dim; j++) {
      x[j] = ab[j].first + (i % segments) * height[j] + height[j] * 0.5;
    }
    result += fn(x);
  }

  for (int i = 0; i < integral_dim; i++) {
    result *= height[i];
  }
  return result;
}

double ParallelTrapezoidal(int segments, ABs ab, const func fn) {
  int world_size, proc_rank;
  MPI_Comm_size(MPI_COMM_WORLD, &world_size);
  MPI_Comm_rank(MPI_COMM_WORLD, &proc_rank);

  int integral_dim = static_cast<int>(ab.size());
  vector<double> height(segments);
  int it_all = 1;

  for (int i = 0; i < integral_dim; i++) {
    height[i] = (ab[i].second - ab[i].first) / segments;
    it_all *= segments;
  }

  int delta = it_all / world_size;

  int start = 0;
  if (proc_rank != 0) start = proc_rank * delta + it_all % world_size;
  int end = (proc_rank + 1) * delta + it_all % world_size;

  vector<double> x(integral_dim);
  double local_result = 0.0;
  for (int j = start; j < end; j++) {
    for (int i = 0; i < integral_dim; i++) {
      x[i] = ab[i].first + (j % segments) * height[i] + height[i] * 0.5;
    }
    local_result += fn(x);
  }

  double global_result = 0.0;
  MPI_Reduce(&local_result, &global_result, 1, MPI_DOUBLE, MPI_SUM, 0,
             MPI_COMM_WORLD);

  if (proc_rank == 0) {
    for (int i = 0; i < integral_dim; i++) {
      global_result *= height[i];
    }
  }

  return global_result;
}
\end{lstlisting}
main.cpp
\begin{lstlisting}
// Copyright 2021 Votyakova Daria
#include <gtest/gtest.h>

#include <cmath>
#include <gtest-mpi-listener.hpp>

#include "./trapezoidal.h"

using std::cout;
using std::endl;

const double error = 0.01;

const function<double(vector<double>)> f1 = [](vector<double> vec) {
  double x = vec[0];
  return x * x + 10 * x - 5;
};

const function<double(vector<double>)> f2 = [](vector<double> vec) {
  double x = vec[0];
  double y = vec[1];
  return x - sin(y) + x * x;
};

const function<double(vector<double>)> f3 = [](vector<double> vec) {
  double x = vec[0];
  double y = vec[1];
  return y * y * y - 10 * cos(x);
};

const function<double(vector<double>)> f4 = [](vector<double> vec) {
  double x = vec[0];
  double y = vec[1];
  double z = vec[2];
  return x * x * x + sin(y) * 2 - 10 * z;
};

const function<double(vector<double>)> f5 = [](vector<double> vec) {
  double x = vec[0];
  double y = vec[1];
  double z = vec[2];
  return (x + y + z) * 10 * log10(z);
};

TEST(Trapezoidal_method_MPI, TEST_1) {
  int rank;
  MPI_Comm_rank(MPI_COMM_WORLD, &rank);

  ABs abs = {{0, 2}};
  int segments = 100;

  double parallel_integral = ParallelTrapezoidal(segments, abs, f1);

  if (rank == 0) {
    double sequential_integral = SequentialTrapezoidal(segments, abs, f1);

    ASSERT_NEAR(sequential_integral, parallel_integral, error);
  }
}

TEST(Trapezoidal_method_MPI, TEST_2) {
  int rank;
  MPI_Comm_rank(MPI_COMM_WORLD, &rank);

  ABs abs = {{0, 2}, {-1, 5}};
  int segments = 50;

  double parallel_integral = ParallelTrapezoidal(segments, abs, f2);

  if (rank == 0) {
    double sequential_integral = SequentialTrapezoidal(segments, abs, f2);

    ASSERT_NEAR(sequential_integral, parallel_integral, error);
  }
}

TEST(Trapezoidal_method_MPI, TEST_3) {
  int rank;
  MPI_Comm_rank(MPI_COMM_WORLD, &rank);

  ABs abs = {{0, 2}, {1, 3}};
  int segments = 100;

  double parallel_integral = ParallelTrapezoidal(segments, abs, f3);

  if (rank == 0) {
    double sequential_integral = SequentialTrapezoidal(segments, abs, f3);

    ASSERT_NEAR(sequential_integral, parallel_integral, error);
  }
}

TEST(Trapezoidal_method_MPI, TEST_4) {
  int rank;
  MPI_Comm_rank(MPI_COMM_WORLD, &rank);

  ABs abs = {{0, 2}, {1, 3}, {-2, 3}};
  int segments = 10;

  double parallel_integral = ParallelTrapezoidal(segments, abs, f4);

  if (rank == 0) {
    double sequential_integral = SequentialTrapezoidal(segments, abs, f4);

    ASSERT_NEAR(sequential_integral, parallel_integral, error);
  }
}

TEST(Trapezoidal_method_MPI, TEST_5) {
  int rank;
  MPI_Comm_rank(MPI_COMM_WORLD, &rank);

  ABs abs = {{0, 2}, {6, 7}, {5, 10}};
  int segments = 10;

  double start, end;
  if (rank == 0) start = MPI_Wtime();
  double parallel_integral = ParallelTrapezoidal(segments, abs, f5);
  if (rank == 0) end = MPI_Wtime();

  if (rank == 0) {
    double ptime = end - start;
    cout << "P time: " << ptime << endl;

    start = MPI_Wtime();
    double sequential_integral = SequentialTrapezoidal(segments, abs, f5);
    end = MPI_Wtime();

    double stime = end - start;
    cout << "S time: " << stime << endl;
    cout << "Speedup: " << stime / ptime << endl;

    ASSERT_NEAR(sequential_integral, parallel_integral, error);
  }
}

int main(int argc, char** argv) {
  ::testing::InitGoogleTest(&argc, argv);
  MPI_Init(&argc, &argv);

  ::testing::AddGlobalTestEnvironment(new GTestMPIListener::MPIEnvironment);
  ::testing::TestEventListeners& listeners =
      ::testing::UnitTest::GetInstance()->listeners();

  listeners.Release(listeners.default_result_printer());
  listeners.Release(listeners.default_xml_generator());

  listeners.Append(new GTestMPIListener::MPIMinimalistPrinter);
  return RUN_ALL_TESTS();
}
\end{lstlisting}

\end{document}