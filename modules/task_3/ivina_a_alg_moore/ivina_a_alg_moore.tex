\documentclass{report}

\usepackage[warn]{mathtext}
\usepackage[T2A]{fontenc}
\usepackage[utf8]{luainputenc}
\usepackage[english, russian]{babel}
\usepackage[pdftex]{hyperref}
\usepackage{tempora}
\usepackage[12pt]{extsizes}
\usepackage{listings}
\usepackage{color}
\usepackage{geometry}
\usepackage{enumitem}
\usepackage{multirow}
\usepackage{graphicx}
\usepackage{indentfirst}
\usepackage{amsmath}

\geometry{a4paper,top=2cm,bottom=2cm,left=2.5cm,right=1.5cm}
\setlength{\parskip}{0.5cm}
\setlist{nolistsep, itemsep=0.3cm,parsep=0pt}

\usepackage{listings}
\lstset{language=C++,
        basicstyle=\footnotesize,
		keywordstyle=\color{blue}\ttfamily,
		stringstyle=\color{red}\ttfamily,
		commentstyle=\color{green}\ttfamily,
		morecomment=[l][\color{red}]{\#}, 
		tabsize=4,
		breaklines=true,
  		breakatwhitespace=true,
  		title=\lstname,       
}

\makeatletter
\renewcommand\@biblabel[1]{#1.\hfil}
\makeatother

\begin{document}

\begin{titlepage}

\begin{center}
Министерство науки и высшего образования Российской Федерации
\end{center}

\begin{center}
Федеральное государственное автономное образовательное учреждение высшего образования \\
Национальный исследовательский Нижегородский государственный университет им. Н.И. Лобачевского
\end{center}

\begin{center}
Институт информационных технологий, математики и механики
\end{center}

\vspace{4em}

\begin{center}
\textbf{\LargeОтчет по лабораторной работе} \\
\end{center}
\begin{center}
\textbf{\Large«Поиск кратчайших путей из одной вершины (алгоритм Мура)»} \\
\end{center}

\vspace{4em}

\newbox{\lbox}
\savebox{\lbox}{\hbox{text}}
\newlength{\maxl}
\setlength{\maxl}{\wd\lbox}
\hfill\parbox{7cm}{
\hspace*{5cm}\hspace*{-5cm}\textbf{Выполнила:} \\ студентка группы 381906-3 \\ Ивина А. С.\\
\\
\hspace*{5cm}\hspace*{-5cm}\textbf{Проверил:}\\ доцент кафедры МОСТ, \\ кандидат технических наук \\ Сысоев А. В.\\
}
\vspace{\fill}

\begin{center} Нижний Новгород \\ 2021 \end{center}

\end{titlepage}

\setcounter{page}{2}

% Содержание
\tableofcontents
\newpage

% Введение
\section*{Введение}
\addcontentsline{toc}{section}{Введение}
\par Алгоритм Мура является усовершенствованием алгоритма Беллмана – Форда, вычисляющего кратчайшие пути от одной вершины. Данная работа направлена на изучение этого алгоритма.
\newpage

% Постановка задачи
\section*{Постановка задачи}
\addcontentsline{toc}{section}{Постановка задачи}
\par Разработать и реализовать последовательный алгоритм Мура. Реализовать алгоритм Мура с использованием MPI. Написать тесты, демонстрирующие использование данного алгоритма, а также сравнить две реализации. Для написания программы используется язык C++ и MPI.
\newpage

% Описание алгоритма
\section*{Описание алгоритма}
\addcontentsline{toc}{section}{Описание алгоритма}
\par Создается граф, состоящий из $V$ вершин, $W$ - вес ребра, соединяющего две вершины, $S$ - вершина, относительно которой высчитывается расстояние до других вершин. Основной алгоритм:
\par Создается массив $res$, в котором будет храниться кратчайшее расстояние от вершины $S$ до всех других вершин. Расстояние от вершины до самой себя равно нулю. Далее создается очередь, в которую помещаются исходная вершина и затем, проходя по всем ребрам, вычисляется наименьший путь в каждую вершину следующим образом: из начала очереди берется вершина, удаляется оттуда, проводится проверка – если $res[new]>res[prev]+W(new,prev)$, то $res[new] = res[prev]+W(new,prev)$, новые вершины добавляются в очередь. В результате получается готовый массив, содержащий кратчайшие пути. 
\par Для удобства вывода результата можно использовать двумерный массив, хранящий номер вершины и расстояние до нее.
\newpage

% Описание схемы распараллеливания
\section*{Описание схемы распараллеливания}
\addcontentsline{toc}{section}{Описание схемы распараллеливания}
\par После создания очереди и массива для результата процесс с рангом 0 выступает в роли слушателя, если процессов больше одного. Он не выполняет основной алгоритм, вместо этого он создает временный массив и ожидает результата от других процессов. Получая результат от одного из процессов он кладет массив с кратчайшими путями в массив массивов. Таким образом, после получения результата от всех процессов получается массив, содержащий эти результаты.
\par Остальные процессы выполняют основной алгоритм, отсылая результат нулевому процессу.
\newpage

% Описание программной реализации
\section*{Описание программной реализации}
\addcontentsline{toc}{section}{Программная реализация}
Программа состоит из следующих файлов: alg\_moore.h, alg\_moore.cpp и main.cpp.
\par alg\_moore.h содержит прототип последовательной функци, параллельной функции и функции генерации рандомного числа в указанном диапазоне.
\par Функция, реализующая последовательный алгоритм Мура, где первый параметр – граф, представляющий собой вектор векторов, содержащий переменные типа pair, второй – вершина, относительно которой высчитываются пути и третий – количество вершин:
\begin{lstlisting}
std::vector<int> shortestPathFaster(const std::vector<std::vector<std::pair<int, int>>>
&graph, const int S, const int V);
\end{lstlisting}
\par Функция, реализующая параллельный алгоритм Мура:
\begin{lstlisting}
std::vector<std::vector<int>> par_shortestPathFaster(std::vector<std::vector<std::pair<int, int>>> graph, const int S, const int V);
\end{lstlisting}
\par Входные параметры идентичны.
\par Функция, генерирующая рандомное число в диапазоне:
\begin{lstlisting}
int randomValue(int min, int max);
\end{lstlisting}
\par alg\_moore.cpp содержит реализации указанных функций. 
\par main.cpp содержит тесты для проверки алгоритмов.
\newpage

% Подтверждение корректности
\section*{Подтверждение корректности}
\addcontentsline{toc}{section}{Подтверждение корректности}
Проверка корректности осуществляется с помощью Google Test. В каждом из 5 тестов создаются графы разных размеров и в конце сравниваются результаты работы двух алгоритмов. 
\par Выполнялись следующие тесты:
\begin{itemize}
  \item Граф с 10 вершинами.
  \item Граф с 20 вершинами.
  \item Граф с двумя вершинами.
  \item Граф с 30 вершинами.
  \item Граф с 50 вершинами.
\end{itemize}
\par Успешное прохождение тестов означает, что алгоритмы работают корректно.
\newpage

% Результаты экспериментов
\section*{Результаты экспериментов}
\addcontentsline{toc}{section}{Результаты экспериментов}
Получились следующие результаты:
\par Результаты экспериментов представлены в Таблице 1

\begin{table}[!h]
\caption{Результаты вычислительных экспериментов в тесте 4}
\centering
\begin{tabular}{| p{2cm} | p{3cm} | p{4cm} | p{2cm} |}
\hline
Количество процессов & Время работы последовательного алгоритма (в секундах) & Время работы параллельного алгоритма (в секундах)  \\[5pt]
\hline
1        & 0.00107        & 0.00068      \\
2        & 0.00016        & 0.00054      \\
4        & 0.00012        & 0.00096      \\
6        & 0.0002         & 0.0027       \\
10       & 0.0001         & 0.0068       \\

\hline
\end{tabular}
\end{table}
\newpage

% Вывод
\section*{Вывод}
\addcontentsline{toc}{section}{Вывод}
Из полученных данных можно сделать вывод, что неэффективное использование распараллеливания может сильно увеличить время работы программы. В данной программе целью была проверка того, корректно ли работают все процессы, однако при сравнении такого способа реализации параллельной программы с последовательным выполнением получилось, что распараллеленный алгоритм тратит больше времени на выполнение.Поэтому эффективность параллельного алгоритма зависит от того, как был реализован алгоритм.
\newpage

% Литература
\section*{Литература}
\addcontentsline{toc}{section}{Литература}
\begin{enumerate}
\item Wikipedia \newline URL: https://en.wikipedia.org/wiki/Shortest\_Path\_Faster\_Algorithm
\item Wikibgu \newline URL: https://wikibgu.ru/wiki/Shortest\_Path\_Faster\_Algorithm
\end{enumerate} 
\newpage

% Приложение
\section*{Приложение}
\addcontentsline{toc}{section}{Приложение}

alg\_moore.h
\begin{lstlisting}
// Copyright 2021 Ivina Anastasiya
#ifndef MODULES_TASK_3_IVINA_A_ALG_MOORE_ALG_MOORE_H_
#define MODULES_TASK_3_IVINA_A_ALG_MOORE_ALG_MOORE_H_

#include <vector>
#include <utility>

int randomValue(int min, int max);
std::vector<int> shortestPathFaster(
    const std::vector<std::vector<std::pair<int, int>>> &graph, const int S, const int V);
std::vector<std::vector<int>> par_shortestPathFaster(
    const std::vector<std::vector<std::pair<int, int>>> &graph, const int S, const int V);

#endif  // MODULES_TASK_3_IVINA_A_ALG_MOORE_ALG_MOORE_H_

\end{lstlisting}
alg\_moore.cpp
\begin{lstlisting}
// Copyright 2021 Ivina Anastasiya
#include <mpi.h>
#include <vector>
#include <queue>
#include <random>
#include "../../../modules/task_3/ivina_a_alg_moore/alg_moore.h"


int randomValue(int min, int max) {
  std::random_device rd;
  std::mt19937 gen(rd());
  std::uniform_int_distribution<> range(min, max);
  return range(rd);
}

std::vector<int> shortestPathFaster(
    const std::vector<std::vector<std::pair<int, int>>> &graph, const int S,
    const int V) {
  std::vector<int> res(V + 1);
  std::vector<bool> Ver_queue(V + 1);

  for (int i = 0; i <= V; i++) {
    res[i] = INT_MAX;
  }
  res[S] = 0;
  std::queue<int> vertexQ;
  vertexQ.push(S);
  Ver_queue[S] = true;
  int step_counter = 0;

  while (!vertexQ.empty() && step_counter < 100) {
    step_counter++;

    int up = vertexQ.front();
    vertexQ.pop();
    Ver_queue[up] = false;

    for (int i = 0; i < static_cast<int>(graph[up].size()); i++) {
      int currentVer = graph[up][i].first;
      int weight = graph[up][i].second;
      if (res[currentVer] > res[up] + weight) {
        res[currentVer] = res[up] + weight;

        if (!Ver_queue[currentVer]) {
          vertexQ.push(currentVer);
          Ver_queue[currentVer] = true;
        }
      }
    }
  }
  return res;
}

std::vector<std::vector<int>> par_shortestPathFaster(
    const std::vector<std::vector<std::pair<int, int>>> &graph, const int S,
    const int V) {
  int size, rank;
  MPI_Comm_size(MPI_COMM_WORLD, &size);
  MPI_Comm_rank(MPI_COMM_WORLD, &rank);

  std::vector<int> res(V + 1);
  std::vector<bool> Ver_queue(V + 1);

  res[S] = 0;

  std::queue<int> vertexQ;
  vertexQ.push(S);
  Ver_queue[S] = true;
  std::vector<std::vector<int>> buf;
  int step_counter = 0;

  if ((size == 1) || (rank != 0)) {
    while (!vertexQ.empty() && step_counter < 100) {
      step_counter++;

      int up = vertexQ.front();
      vertexQ.pop();
      Ver_queue[up] = false;

      for (int i = 0; i < static_cast<int>(graph[up].size()); i++) {
        int currentVer = graph[up][i].first;
        int weight = graph[up][i].second;
        if (res[currentVer] > res[up] + weight) {
          res[currentVer] = res[up] + weight;

          if (!Ver_queue[currentVer]) {
            vertexQ.push(currentVer);
            Ver_queue[currentVer] = true;
          }
        }
      }
    }
    if (size == 1) {
      buf.push_back(res);
    } else {
      MPI_Send(res.data(), res.size(), MPI_INT, 0, 0, MPI_COMM_WORLD);
    }
  } else {
    for (int i = 0; i < size - 1; i++) {
      std::vector<int> temp;

      for (int j = 0; j < V + 1; j++) {
        temp.push_back(0);
      }
      MPI_Recv(temp.data(), temp.size(), MPI_INT, MPI_ANY_SOURCE, 0,
               MPI_COMM_WORLD, MPI_STATUSES_IGNORE);
      buf.push_back(temp);
    }
  }
  return buf;
}

\end{lstlisting}
main.cpp
\begin{lstlisting}
// Copyright 2021 Ivina Anastasiya
#include <gtest/gtest.h>
#include <vector>
#include <iostream>
#include "./alg_moore.h"
#include <gtest-mpi-listener.hpp>


TEST(alg_moore, vertex_num_10) {
  int rank;
  int Vertex = 10;
  std::vector<std::vector<std::pair<int, int>>> graph(Vertex + 1);
  MPI_Comm_rank(MPI_COMM_WORLD, &rank);
  int root = 1;
  int a = 0;
  int b = 0;
  std::vector<std::pair<int, int>> temp;
  if (rank == 0) {
    for (int i = 0; i < 15; i++) {
      temp.resize(i + 1);
      a = randomValue(1, 10);
      b = randomValue(1, 10);
      for (const auto &j : temp) {
        while ((j.first == a && j.second == b) ||
               (j.first == b && j.second == a) || (a == b)) {
          a = randomValue(1, 10);
          b = randomValue(1, 10);
        }
      }
      temp[i].first = a;
      temp[i].second = b;
      graph[a].push_back({b, randomValue(1, 10)});
    }
  }
  int to_send_size = 0;
  int to_send_vertex_num = 0;

  struct vertex {
    int id;
    int to;
    int w;
  };
  std::vector<vertex> to_send;
  if (rank == 0) {
    for (auto i = 0; i < Vertex + 1; ++i) {
      for (const auto &n : graph[i]) {
        to_send.push_back({i, n.first, n.second});
      }
    }
    to_send_size = to_send.size();
    to_send_vertex_num = graph.size();
  }
  MPI_Bcast(&to_send_size, 1, MPI_INT, 0, MPI_COMM_WORLD);
  MPI_Bcast(&to_send_vertex_num, 1, MPI_INT, 0, MPI_COMM_WORLD);

  if (rank != 0) {
    graph.clear();
    to_send.clear();
    graph.resize(to_send_vertex_num);
  }

  for (int i = 0; i < to_send_size; ++i) {
    vertex ver{};
    if (rank == 0) {
      ver = to_send[i];
    }
    MPI_Bcast(&ver, 3, MPI_INT, 0, MPI_COMM_WORLD);
    if (rank != 0) {
      graph[ver.id].push_back({ver.to, ver.w});
    }
  }
  to_send.clear();

  double start = MPI_Wtime();
  auto result = par_shortestPathFaster(graph, root, Vertex);
  double end = MPI_Wtime();

  if (rank == 0) {
    double par_time = end - start;
    double start = MPI_Wtime();
    auto result2 = shortestPathFaster(graph, root, Vertex);
    double end = MPI_Wtime();
    double seq_time = end - start;
    std::cout << "par_time = " << par_time << std::endl;
    std::cout << "seq_time = " << seq_time << std::endl;

    if (result.empty() == false) {
      for (const auto paralel : result) {
        EXPECT_EQ(paralel.size(), result2.size());
        for (int i = 0; i < static_cast<int>(paralel.size()); ++i) {
          EXPECT_EQ(paralel[i], result2[i]);
        }
      }
    }
  }
}

TEST(alg_moore, vertex_num_20) {
  int rank;
  int Vertex = 20;
  std::vector<std::vector<std::pair<int, int>>> graph(Vertex + 1);
  MPI_Comm_rank(MPI_COMM_WORLD, &rank);
  int root = 1;
  int a = 0;
  int b = 0;
  std::vector<std::pair<int, int>> temp;
  if (rank == 0) {
    for (int i = 0; i < 25; i++) {
      temp.resize(i + 1);
      a = randomValue(1, 20);
      b = randomValue(1, 20);
      for (const auto j : temp) {
        while ((j.first == a && j.second == b) ||
               (j.first == b && j.second == a) || (a == b)) {
          a = randomValue(1, 20);
          b = randomValue(1, 20);
        }
      }
      temp[i].first = a;
      temp[i].second = b;
      graph[a].push_back({b, randomValue(1, 10)});
    }
  }
  int to_send_size = 0;
  int to_send_vertex_num = 0;

  struct vertex {
    int id;
    int to;
    int w;
  };
  std::vector<vertex> to_send;
  if (rank == 0) {
    for (int i = 0; i < Vertex + 1; ++i) {
      for (const auto n : graph[i]) {
        to_send.push_back({i, n.first, n.second});
      }
    }
    to_send_size = to_send.size();
    to_send_vertex_num = graph.size();
  }
  MPI_Bcast(&to_send_size, 1, MPI_INT, 0, MPI_COMM_WORLD);
  MPI_Bcast(&to_send_vertex_num, 1, MPI_INT, 0, MPI_COMM_WORLD);

  if (rank != 0) {
    graph.clear();
    to_send.clear();
    graph.resize(to_send_vertex_num);
  }

  for (int i = 0; i < to_send_size; ++i) {
    vertex ver{};
    if (rank == 0) {
      ver = to_send[i];
    }
    MPI_Bcast(&ver, 3, MPI_INT, 0, MPI_COMM_WORLD);
    if (rank != 0) {
      graph[ver.id].push_back({ver.to, ver.w});
    }
  }
  to_send.clear();

  double start = MPI_Wtime();
  auto result = par_shortestPathFaster(graph, root, Vertex);
  double end = MPI_Wtime();

  if (rank == 0) {
    double par_time = end - start;
    double start = MPI_Wtime();
    auto result2 = shortestPathFaster(graph, root, Vertex);
    double end = MPI_Wtime();
    double seq_time = end - start;
    std::cout << "par_time = " << par_time << std::endl;
    std::cout << "seq_time = " << seq_time << std::endl;

    if (result.empty() == false) {
      for (const auto paralel : result) {
        EXPECT_EQ(paralel.size(), result2.size());
        for (int i = 0; i < static_cast<int>(paralel.size()); ++i) {
          EXPECT_EQ(paralel[i], result2[i]);
        }
      }
    }
  }
}

TEST(alg_moore, vertex_num_2) {
  int rank;
  int Vertex = 2;
  std::vector<std::vector<std::pair<int, int>>> graph(Vertex + 1);
  MPI_Comm_rank(MPI_COMM_WORLD, &rank);
  int root = 1;
  if (rank == 0) {
    for (int i = 0; i < 1; i++) {
      int a = 1;
      int b = 2;
      graph[a].push_back({b, randomValue(1, 10)});
    }
  }
  int to_send_size = 0;
  int to_send_vertex_num = 0;

  struct vertex {
    int id;
    int to;
    int w;
  };
  std::vector<vertex> to_send;
  if (rank == 0) {
    for (int i = 0; i < Vertex + 1; ++i) {
      for (const auto n : graph[i]) {
        to_send.push_back({i, n.first, n.second});
      }
    }
    to_send_size = to_send.size();
    to_send_vertex_num = graph.size();
  }
  MPI_Bcast(&to_send_size, 1, MPI_INT, 0, MPI_COMM_WORLD);
  MPI_Bcast(&to_send_vertex_num, 1, MPI_INT, 0, MPI_COMM_WORLD);

  if (rank != 0) {
    graph.clear();
    to_send.clear();
    graph.resize(to_send_vertex_num);
  }

  for (int i = 0; i < to_send_size; ++i) {
    vertex ver{};
    if (rank == 0) {
      ver = to_send[i];
    }
    MPI_Bcast(&ver, 3, MPI_INT, 0, MPI_COMM_WORLD);
    if (rank != 0) {
      graph[ver.id].push_back({ver.to, ver.w});
    }
  }
  to_send.clear();

  double start = MPI_Wtime();
  auto result = par_shortestPathFaster(graph, root, Vertex);
  double end = MPI_Wtime();

  if (rank == 0) {
    double par_time = end - start;
    double start = MPI_Wtime();
    auto result2 = shortestPathFaster(graph, root, Vertex);
    double end = MPI_Wtime();
    double seq_time = end - start;
    std::cout << "par_time = " << par_time << std::endl;
    std::cout << "seq_time = " << seq_time << std::endl;

    if (result.empty() == false) {
      for (const auto paralel : result) {
        EXPECT_EQ(paralel.size(), result2.size());
        for (int i = 0; i < static_cast<int>(paralel.size()); ++i) {
          EXPECT_EQ(paralel[i], result2[i]);
        }
      }
    }
  }
}

TEST(alg_moore, vertex_num_30) {
  int rank;
  int Vertex = 30;
  std::vector<std::vector<std::pair<int, int>>> graph(Vertex + 1);
  MPI_Comm_rank(MPI_COMM_WORLD, &rank);
  int root = 1;
  int a = 0;
  int b = 0;
  std::vector<std::pair<int, int>> temp;
  if (rank == 0) {
    for (int i = 0; i < 35; i++) {
      temp.resize(i + 1);
      a = randomValue(1, 30);
      b = randomValue(1, 30);
      for (const auto j : temp) {
        while ((j.first == a && j.second == b) ||
               (j.first == b && j.second == a) || (a == b)) {
          a = randomValue(1, 30);
          b = randomValue(1, 30);
        }
      }
      temp[i].first = a;
      temp[i].second = b;
      graph[a].push_back({b, randomValue(1, 10)});
    }
  }
  int to_send_size = 0;
  int to_send_vertex_num = 0;

  struct vertex {
    int id;
    int to;
    int w;
  };
  std::vector<vertex> to_send;
  if (rank == 0) {
    for (int i = 0; i < Vertex + 1; ++i) {
      for (const auto n : graph[i]) {
        to_send.push_back({i, n.first, n.second});
      }
    }
    to_send_size = to_send.size();
    to_send_vertex_num = graph.size();
  }
  MPI_Bcast(&to_send_size, 1, MPI_INT, 0, MPI_COMM_WORLD);
  MPI_Bcast(&to_send_vertex_num, 1, MPI_INT, 0, MPI_COMM_WORLD);

  if (rank != 0) {
    graph.clear();
    to_send.clear();
    graph.resize(to_send_vertex_num);
  }

  for (int i = 0; i < to_send_size; ++i) {
    vertex ver{};
    if (rank == 0) {
      ver = to_send[i];
    }
    MPI_Bcast(&ver, 3, MPI_INT, 0, MPI_COMM_WORLD);
    if (rank != 0) {
      graph[ver.id].push_back({ver.to, ver.w});
    }
  }
  to_send.clear();

  double start = MPI_Wtime();
  auto result = par_shortestPathFaster(graph, root, Vertex);
  double end = MPI_Wtime();

  if (rank == 0) {
    double par_time = end - start;
    double start = MPI_Wtime();
    auto result2 = shortestPathFaster(graph, root, Vertex);
    double end = MPI_Wtime();
    double seq_time = end - start;
    std::cout << "par_time = " << par_time << std::endl;
    std::cout << "seq_time = " << seq_time << std::endl;

    if (result.empty() == false) {
      for (const auto paralel : result) {
        EXPECT_EQ(paralel.size(), result2.size());
        for (int i = 0; i < static_cast<int>(paralel.size()); ++i) {
          EXPECT_EQ(paralel[i], result2[i]);
        }
      }
    }
  }
}

TEST(alg_moore, vertex_num_50) {
  int rank;
  int Vertex = 50;
  std::vector<std::vector<std::pair<int, int>>> graph(Vertex + 1);
  MPI_Comm_rank(MPI_COMM_WORLD, &rank);
  int root = 1;
  int a = 0;
  int b = 0;
  std::vector<std::pair<int, int>> temp;
  if (rank == 0) {
    for (int i = 0; i < 55; i++) {
      temp.resize(i + 1);
      a = randomValue(1, 50);
      b = randomValue(1, 50);
      for (const auto j : temp) {
        while ((j.first == a && j.second == b) ||
               (j.first == b && j.second == a) || (a == b)) {
          a = randomValue(1, 50);
          b = randomValue(1, 50);
        }
      }
      temp[i].first = a;
      temp[i].second = b;
      graph[a].push_back({b, randomValue(1, 10)});
    }
  }
  int to_send_size = 0;
  int to_send_vertex_num = 0;

  struct vertex {
    int id;
    int to;
    int w;
  };
  std::vector<vertex> to_send;
  if (rank == 0) {
    for (int i = 0; i < Vertex + 1; ++i) {
      for (const auto n : graph[i]) {
        to_send.push_back({i, n.first, n.second});
      }
    }
    to_send_size = to_send.size();
    to_send_vertex_num = graph.size();
  }
  MPI_Bcast(&to_send_size, 1, MPI_INT, 0, MPI_COMM_WORLD);
  MPI_Bcast(&to_send_vertex_num, 1, MPI_INT, 0, MPI_COMM_WORLD);

  if (rank != 0) {
    graph.clear();
    to_send.clear();
    graph.resize(to_send_vertex_num);
  }

  for (int i = 0; i < to_send_size; ++i) {
    vertex ver{};
    if (rank == 0) {
      ver = to_send[i];
    }
    MPI_Bcast(&ver, 3, MPI_INT, 0, MPI_COMM_WORLD);
    if (rank != 0) {
      graph[ver.id].push_back({ver.to, ver.w});
    }
  }
  to_send.clear();

  double start = MPI_Wtime();
  auto result = par_shortestPathFaster(graph, root, Vertex);
  double end = MPI_Wtime();

  if (rank == 0) {
    double par_time = end - start;
    double start = MPI_Wtime();
    auto result2 = shortestPathFaster(graph, root, Vertex);
    double end = MPI_Wtime();
    double seq_time = end - start;
    std::cout << "par_time = " << par_time << std::endl;
    std::cout << "seq_time = " << seq_time << std::endl;

    if (result.empty() == false) {
      for (const auto paralel : result) {
        EXPECT_EQ(paralel.size(), result2.size());
        for (int i = 0; i < static_cast<int>(paralel.size()); ++i) {
          EXPECT_EQ(paralel[i], result2[i]);
        }
      }
    }
  }
}

int main(int argc, char **argv) {
  ::testing::InitGoogleTest(&argc, argv);
  MPI_Init(&argc, &argv);

  ::testing::AddGlobalTestEnvironment(new GTestMPIListener::MPIEnvironment);
  ::testing::TestEventListeners &listeners =
      ::testing::UnitTest::GetInstance()->listeners();

  listeners.Release(listeners.default_result_printer());
  listeners.Release(listeners.default_xml_generator());

  listeners.Append(new GTestMPIListener::MPIMinimalistPrinter);
  return RUN_ALL_TESTS();
}

\end{lstlisting}

\end{document}