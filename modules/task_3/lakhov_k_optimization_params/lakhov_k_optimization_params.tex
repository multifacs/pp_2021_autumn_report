\documentclass{report}

\usepackage[warn]{mathtext}
\usepackage[T2A]{fontenc}
\usepackage[utf8]{luainputenc}
\usepackage[english, russian]{babel}
\usepackage[pdftex]{hyperref}
\usepackage{tempora}
\usepackage[12pt]{extsizes}
\usepackage{listings}
\usepackage{color}
\usepackage{geometry}
\usepackage{enumitem}
\usepackage{multirow}
\usepackage{graphicx}
\usepackage{indentfirst}
\usepackage{amsmath}

\geometry{a4paper,top=2cm,bottom=2cm,left=2.5cm,right=1.5cm}
\setlength{\parskip}{0.5cm}
\setlist{nolistsep, itemsep=0.3cm,parsep=0pt}

\usepackage{listings}
\lstset{language=C++,
        basicstyle=\footnotesize,
        keywordstyle=\color{blue}\ttfamily,
        stringstyle=\color{red}\ttfamily,
        commentstyle=\color{green}\ttfamily,
        morecomment=[l][\color{red}]{\#}, 
        tabsize=4,
        breaklines=true,
        breakatwhitespace=true,
        title=\lstname,       
}

\makeatletter
\renewcommand\@biblabel[1]{#1.\hfil}
\makeatother

\begin{document}

\begin{titlepage}

\begin{center}
Министерство науки и высшего образования Российской Федерации
\end{center}

\begin{center}
Федеральное государственное автономное образовательное учреждение высшего образования \\
Национальный исследовательский Нижегородский государственный университет им. Н.И. Лобачевского
\end{center}

\begin{center}
Институт информационных технологий, математики и механики
\end{center}

\vspace{4em}

\begin{center}
\textbf{\LargeОтчет по лабораторной работе № 3} \\
\end{center}
\begin{center}
\textbf{\Large«Многошаговая схема решения двумерных задач глобальной оптимизации. 
(Распараллеливание по характеристикам).»} \\
\end{center}

\vspace{4em}

\newbox{\lbox}
\savebox{\lbox}{\hbox{text}}
\newlength{\maxl}
\setlength{\maxl}{\wd\lbox}
\hfill\parbox{7cm}{
\hspace*{5cm}\hspace*{-5cm}\textbf{Выполнил:} \\ студент группы 381908-1 \\ Лахов Кирилл\\
\\
\hspace*{5cm}\hspace*{-5cm}\textbf{Преподаватель:}\\ доцент кафедры МОСТ, \\ кандидат технических наук \\ Сысоев А. В.\\
}
\vspace{\fill}

\begin{center} Нижний Новгород \\ 2021 \end{center}

\end{titlepage}

\setcounter{page}{2}

% Содержание
\tableofcontents
\newpage

% Введение
\section*{Введение}
\addcontentsline{toc}{section}{Введение}
\par В широком смысле, глобальная оптимизация  - это раздел численного анализа который занимается проблемами поиска глобального экстремума функции. Как правило решается задача минимизации, так как поиск максимума является эквивалентным.
Одним из широко известных методов для решения задачи многомерной глобальной оптимизации является применение многошаговой схемы редукции размерности. Согласно ей решение многомерной задачи сводится к последовательному решению вложенных одномерных задач.
\newpage

% Постановка задачи
\section*{Постановка задачи}
\addcontentsline{toc}{section}{Постановка задачи}
\par В данной лабораторной работе необходимо реализовать последовательный и параллельный алгоритм могошаговой схемы редукции размерности для двумерных функций. За основу параллельной схемы взять характеристики интервалов. Также требуется провести тестирование работоспособности алгоритма при помощи Google Tests, выявить эффективность парралельного варианта.
\par Параллельный алгоритм должен быть реализован при помощи технологии MPI.
\newpage

\section*{Описание алгоритма}
\addcontentsline{toc}{section}{Описание алгоритма}
Пусть задана функция $\phi(x, y)$. Использование многошаговой схемы редукции размерности сводит задачу поиска минимума к серии одномерных задач:
$$\min\limits_{y' \in D}\phi(y') = \min\limits_{y \in [a_1; b_1]} \min\limits_{x \in [a_2; b_2]}\phi(x, y)$$
Рассмотрим минимизацию одномерной функции $\phi(x)$ по переменной $x$ на интервале $[a, b]$. Предположим что выполнено $n>1$ итераций метода (в качестве начальных точек $x_1, x_2, \ldots, x_k$ можно взять случайные или граничные точки). Тогда новая точка будет получатся по следующему алгоритму:
\par Шаг 1. Перенумеровать точки $x_1, x_2, \ldots, x_k$ так, что
$$ a = x_1 < x_2 \ldots < x_k = b$$
\par Шаг 2. Полагая $z_i = \phi(x_i)$ вычислить величины, при этом $r>1$ является заданным параметром надежности:
$$\mu = \max\limits_{1 \leq i \leq k} \frac{|z_i - z_{i-1}|}{x_i - x_{i-1}}, M = \begin{cases} r\mu, & \mu > 0 \\ 1, & \mu=0 \end{cases} $$

\par Шаг 3. Для каждого интервала $(x_{i-1}, x_i), 1 \leq i \leq k+1$ вычислить характеристику:
$$ R(i) = M(x_i - x_{i-1}) + \frac{(z_i - z_{i-1})^2}{x_i - x_{i-1}} - 2(z_i - z_{i-1})$$
\par Шаг 4. Характеристики $R(i)$ упорядочить в порядке убывания:
$$ R(t_1) \geq R(t_2) \geq \ldots \geq R(t_{k-1})$$
Затем необходимо выбрать наибольшую характеристику с номером интервала $t_1$.
\par Шаг 5. Расчитать новую точку $x_{k+1}$:
$$ x_{k+1} = \frac{x_{t_1} + x_{t_1 - 1}}{2} - \frac{z_{t_1} - z_{t_1 - 1}}{2M}$$
\par Алгоритм останавливается, если для номера $t_1$ выполняется условие $x_{t_1} - x_{t_1 - 1} \leq \varepsilon$, где $\varepsilon>0$ есть заданная точность. В качестве решения задачи выбираются значения:
$$\phi_{min} = \min\limits_{1 \leq i \leq k}\phi(x_i), x_{min} = \arg{\min\limits_{1 \leq i \leq k}\phi(x_i)}$$
\par Для получения минимального значения по двум переменным, вышеприведенный алгоритм запускается для $y$. После вычисления очереденой точки $y_i$ по этому алгоритму решается вложенная задача минимизации по всему интервалу $x$  при постоянном $y_i$. Затем выполнение алгоритма по переменной $y$ продолжается.


\newpage

\section*{Описание схемы распараллеливания}
\addcontentsline{toc}{section}{Описание схемы распараллеливания}
\par Распараллеливание алгоритма осуществляется следующим образом: решение вложенной задачи минимизации по переменной $x$ на основе $n$ наибольших характеристик производится параллельно. Вычисление характеристик для переменной $y$ остается в главном процессе с рангом 0. Соответственно модифицируются шаги 4 и 5 алгоритма по переменной $y$.
\par Шаг 4. Характеристики $R(i)$ упорядочить в порядке убывания:
$$ R(t_1) \geq R(t_2) \geq \ldots \geq R(t_{k-1})$$
Затем необходимо выбрать $n$ наибольших характеристик с номером интервала $t_j$ $1 \leq j \leq n$.
\par Шаг 5. Расчитать новые точки $y_{k+j}, 1 \leq j \leq n$:
$$ y_{k+1} = \frac{y_{t_j} + y_{t_j - 1}}{2} - \frac{z_{t_j} - z_{t_j - 1}}{2M}$$
\par После расчета новых точек, для каждой из них необходимо решить вложенную задачу по $x$, это происходит параллельно. Все дополнительные процессы ожидают некоторую точку $y$ для которой необходимо решить вложенную задачу по всему интервалу $x$.
\par В данном случае для $y$ алгоритм прекращает работу, если хотя бы для одного номера $t_j$, $1 \leq j \leq n$ выполняется $y_{t_j} - y_{t_j - 1} \leq \varepsilon$.
\newpage

\section*{Описание программной реализации}
\addcontentsline{toc}{section}{Описание программной реализации}
Программа состоит из заголовочного файла optimization\_params.h и двух файлов исходного кода optimization\_params.cpp и main.cpp.
\par В заголовочном файле находятся обьявления функций, вспомогательных классов и структур.
\par Структура для хранения точки:
\begin{lstlisting}
struct Point {
    double x;
    double y;
    double z;
};
\end{lstlisting}
\par Классы для упорядочивания точек по для переменных $x$ и $y$:
\begin{lstlisting}
class doubleDimensionChar {
 public:
    const double x;
    const double y;
    const double functionValue;
    doubleDimensionChar(double x_value, double y_value, double f_value)
    : x(x_value), y(y_value), functionValue(f_value) {}
    friend bool operator<(const doubleDimensionChar& left,
                          const doubleDimensionChar& right) {
        return left.y < right.y;
    }
};

class singleDimensionChar {
 public:
    const double x;
    const double functionValue;
    singleDimensionChar(double x_value, double f_value)
    : x(x_value), functionValue(f_value) {}
    friend bool operator<(const singleDimensionChar& left,
                          const singleDimensionChar& right) {
         return left.x < right.x;
    }
};
\end{lstlisting}
\newpage

\par Класс для упорядочивания интервалов по характеристике $R$:
\begin{lstlisting}
class Interval {
 public:
    const double r;
    const double variableFirst;
    const double functionValueFirst;
    const double variableSecond;
    const double functionValueSecond;
    Interval(double r_value, double variable_f,
             double f_f, double variable_s, double f_s)
    : r(r_value), variableFirst(variable_f), functionValueFirst(f_f),
      variableSecond(variable_s),  functionValueSecond(f_s) {}
    friend bool operator<(const Interval& left, const Interval& right) {
        return left.r > right.r;
    }
};
\end{lstlisting}
\par Функция для решения одномерной задачи на итервале. Задаются интервал по $x$, значение $y$, функция для которой необходимо провести расчеты:
\begin{lstlisting}
Point singleDimensionMin(double left_x, double right_x, double const_y,
                         double(*func)(double x, double y));
\end{lstlisting}

\par Функция для последовательного решения задачи. Задаются интервалы по $x$ и $y$, функция для которой необходимо провести расчеты:
\begin{lstlisting}
Point sequentialCalc(double left_x, double right_x,
                     double left_y, double right_y,
                     double(*func)(double x, double y));
\end{lstlisting}

\par Функция парралельного решения задачи. Задаются интервалы по $x$ и $y$, функция для которой необходимо провести расчеты:
\begin{lstlisting}
Point parralelCalc(double left_x, double right_x,
                   double left_y, double right_y,
                   double(*func)(double x, double y));
\end{lstlisting}

\par А так же функции для тестирования f1-f5 и функция сравнения двух точек:
\begin{lstlisting}
double f1(double x, double y);
double f2(double x, double y);
double f3(double x, double y);
double f4(double x, double y);
double f5(double x, double y);
bool comparePoints(const Point& left, const Point& right);
\end{lstlisting}

\par В файле исходного кода optimization\_params.cpp содержится реализация функций,
объявленных в заголовочном файле. В файле исходного кода main.cpp содержатся тесты для
проверки корректности программы.


\newpage

\section*{Корректность алгоритма}
\addcontentsline{toc}{section}{Подтверждение корректности}
Для подтверждения корректности работы данной программы с помощью  Google Tests было разработано 6 тестов. В каждом из тестов для предопределенных функций f1-f5 решается задача с помощью параллельного и последовательного метода. Затем результаты вычислений сравниваются.
\par Программа была протестирована на 5 различных функциях и 10 вариантах количества процессов. Тесты были пройдены успешно, что подтверждает корректность алгоритмов.

\newpage

% Результатов экспериментов
\section*{Результаты экспериментов }
\addcontentsline{toc}{section}{Результаты экспериментов}
Эксперименты проводились для функций:
$$f_1 = (y-1)^2 + x^2$$
$$f_2 = 4 + \sqrt[3]{(x^2+y^2)^2}$$
$$f_3 = x + 4y - 2\log_{10}xy - 3\log_{10}y$$
$$f_4 = x^2 - 2xy + y^3$$
$$f_5 = x^2 + 3xy + 4y^2$$
\par Для каждой функции был написан тест на некоторых интервалах по $x$ и $y$. Номер теста соответствует номеру функции, Тест№6 проверяет первую функцию на большем интервале.
\begin{table}[!h]
\caption{Результаты вычислительных экспериментов на для тестов на 4 процессах. }
\centering
\begin{tabular}{| p{2cm} | p{3cm} | p{4cm} | p{2cm} |}
\hline
Тест & Время работы последовательного алгоритма (в секундах) & Время работы параллельного алгоритма (в секундах) & Ускорение  \\[5pt]
\hline
1        & 0.691909       & 0.362557      & 1.90841       \\
2        & 0.621564       & 0.347626      & 1.78803       \\
3        & 0.0633535      & 0.036273      & 1.74658       \\
4        & 0.475922       & 0.243701      & 1.95289       \\
5        & 0.438491       & 0.214224      & 2.04688       \\
6        & 1.70031        & 0.900267      & 1.88868       \\

\hline
\end{tabular}
\end{table}

\begin{table}[!h]
\caption{Результаты вычислительных экспериментов на для тестов на 8 процессах. }
\centering
\begin{tabular}{| p{2cm} | p{3cm} | p{4cm} | p{2cm} |}
\hline
Тест & Время работы последовательного алгоритма (в секундах) & Время работы параллельного алгоритма (в секундах) & Ускорение  \\[5pt]
\hline
1        & 0.68169        & 0.248065      & 2.74803       \\
2        & 0.623691       & 0.222094      & 2.80823       \\
3        & 0.0638146      & 0.0318189     & 2.00556       \\
4        & 0.478231       & 0.159591      & 2.99661       \\
5        & 0.439277       & 0.161346      & 2.72258       \\
6        & 1.75251        & 0.619884      & 2.82716       \\
\hline
\end{tabular}
\end{table}
\par Успешное прохождение тестов и значительное ускорение работы программы подтверждает корректность реализованного алгоритма.
\newpage

% Выводы из результатов экспериментов
\section*{Выводы из результатов экспериментов }
\addcontentsline{toc}{section}{Выводы из результатов экспериментов}
В результате проведенных тестов можно сделать вывод, что параллельный вариант алгоритма работает значительно быстрее последовательного. Увеличение числа процессов повышает скорость выполнения программы, это можно обьяснить увеличением числа выбираемых характеристик и соответствующих им интервалов.
Также можно заметить что увеличение области поиска минимума делает задачу более трудоемкой, что подчеркивает необходимость использования парралельного алгоритма.

\section*{Заключение}
\addcontentsline{toc}{section}{Заключение}
Таким образом, в рамках данной лабораторной работы были реализованы последовательный и парралельный алгоритмы решения задачи глобальной оптимизации с применением многошаговой схемы редукции размерности. Проведенные эксперименты показали корректность и эффективность реализованной программы.
\newpage


\section*{Литература}
\addcontentsline{toc}{section}{Литература}
\begin{enumerate}
\item В.П. Гергель,  Многомерная многоэкстремальная оптимизация на основе адаптивной многошаговой редукции размерности /  В.П. Гергель,  В.А. Гришагин. – Нижний Новгород : Вестник Нижегородского университета им. Н.И. Лобачевского, 2010. – 170 с.
\item К.А. Баркалов,  Решение задач глобальной оптимизации на графических ускорителях /  К.А. Баркалов,  И.Г. Лебедев. – Нижний Новгород : Нижегородский государственный университет им. Н.И. Лобачевского, 2016. – 11 с.
\item А.В. Сысоев,  MPI-реализация блочной многошаговой схемы параллельного решения задач глобальной оптимизации /  А.В. Сысоев,  В.П. Гергель. – Нижний Новгород : Нижегородский государственный университет им. Н.И. Лобачевского, 2015. – 9 с.
\end{enumerate} 
\newpage


\section*{Приложение}
\addcontentsline{toc}{section}{Приложение}
optimization\_params.h
\begin{lstlisting}
// Copyright 2021 Lakhov Kirill
#ifndef MODULES_TASK_3_LAKHOV_K_OPTIMIZATION_PARAMS_OPTIMIZATION_PARAMS_H_
#define MODULES_TASK_3_LAKHOV_K_OPTIMIZATION_PARAMS_OPTIMIZATION_PARAMS_H_
#include <mpi.h>
#include <cmath>
#include <set>
#include <vector>
#include <limits>

double f1(double x, double y);
double f2(double x, double y);
double f3(double x, double y);
double f4(double x, double y);
double f5(double x, double y);


struct Point {
    double x;
    double y;
    double z;
};

Point sequentialCalc(double left_x, double right_x,
                     double left_y, double right_y,
                     double(*func)(double x, double y));

Point parralelCalc(double left_x, double right_x,
                   double left_y, double right_y,
                   double(*func)(double x, double y));

class Interval {
 public:
    const double r;
    const double variableFirst;
    const double functionValueFirst;
    const double variableSecond;
    const double functionValueSecond;
    Interval(double r_value, double variable_f,
             double f_f, double variable_s, double f_s)
    : r(r_value), variableFirst(variable_f), functionValueFirst(f_f),
      variableSecond(variable_s),  functionValueSecond(f_s) {}
    friend bool operator<(const Interval& left, const Interval& right) {
        return left.r > right.r;
    }
};

class doubleDimensionChar {
 public:
    const double x;
    const double y;
    const double functionValue;
    doubleDimensionChar(double x_value, double y_value, double f_value)
    : x(x_value), y(y_value), functionValue(f_value) {}
    friend bool operator<(const doubleDimensionChar& left,
                          const doubleDimensionChar& right) {
        return left.y < right.y;
    }
};

class singleDimensionChar {
 public:
    const double x;
    const double functionValue;
    singleDimensionChar(double x_value, double f_value)
    : x(x_value), functionValue(f_value) {}
    friend bool operator<(const singleDimensionChar& left,
                          const singleDimensionChar& right) {
         return left.x < right.x;
    }
};

Point singleDimensionMin(double left_x, double right_x, double const_y,
                         double(*func)(double x, double y));

bool comparePoints(const Point& left, const Point& right);

#endif  // MODULES_TASK_3_LAKHOV_K_OPTIMIZATION_PARAMS_OPTIMIZATION_PARAMS_H_
\end{lstlisting}

optimization\_params.cpp
\begin{lstlisting}
// Copyright 2021 Lakhov Kirill
#include "../../../modules/task_3/lakhov_k_optimization_params/optimization_params.h"

double f1(double x, double y) {
    return pow(y - 1, 2) + pow(x, 2);
}

double f2(double x, double y) {
    return 4 + pow(pow(x, 2) + pow(y, 2), 2.0 / 3);
}

double f3(double x, double y) {
    return x + 4 * y - 2 * log10(x*y) - 3 * log10(y);
}

double f4(double x, double y) {
    return pow(x, 2) - 2*x*y + pow(y, 3);
}

double f5(double x, double y) {
    return pow(x, 2) + 3*x*y + 4*pow(y, 2);
}

Point singleDimensionMin(double left_x, double right_x, double const_y,
                         double(*func)(double x, double y)) {
    int n_max_value = 1000;
    double eps_in_func = 0.01;
    double r = 2;

    Point result;
    std::set<singleDimensionChar> set;


    double left_func_value = func(left_x, const_y);
    set.insert(singleDimensionChar(left_x, left_func_value));

    result.x = left_x;
    result.y = const_y;
    result.z = left_func_value;

    double right_func_value = func(right_x, const_y);
    set.insert(singleDimensionChar(right_x, right_func_value));

    if (result.z > right_func_value) {
        result.x = right_x;
        result.z = right_func_value;
    }

    int k = 2;
    auto maxRiter = set.begin();
    auto r_i_value_previous_max = set.begin();
    double mu, mu_current;
    double M;
    bool st_flag = false;
    while (!st_flag && k < n_max_value) {
        // calc mu and M
        mu = -1;
        auto i_value = set.begin();
        i_value++;
        auto i_previous_value = set.begin();
        while (i_value != set.end()) {
            mu_current = std::abs(static_cast<double>(
            (i_value->functionValue - i_previous_value->functionValue) /
            (i_value->x - i_previous_value->x)));
            if (mu_current > mu) {mu = mu_current;}
            i_value++; i_previous_value++;
        }
        if (mu > 0) {
            M = r * mu;
        } else {
            M = 1;
        }

        i_value = set.begin();
        i_value++;
        i_previous_value = set.begin();
        double R, current_r;
        R = std::numeric_limits<double>::lowest();;
        while (i_value != set.end()) {
            double delta_x = (i_value->x - i_previous_value->x);
            double a = M * delta_x;
            double delta_f = i_value->functionValue -
                                    i_previous_value->functionValue;
            double b = pow(delta_f, 2) / a;
            double c = 2 * (delta_f);
            current_r = a + b - c;
            if (current_r > R) {
                R = current_r;
                maxRiter = i_value;
                r_i_value_previous_max = i_previous_value;
            }
            i_value++; i_previous_value++;
        }
        k++;
        double new_x = 0.5 * (maxRiter->x + r_i_value_previous_max->x) -
        ((maxRiter->functionValue - r_i_value_previous_max->functionValue) /
                                                                     (2 * M));
        double new_func_value = func(new_x, const_y);
        set.insert(singleDimensionChar(new_x, new_func_value));
        if (result.z > new_func_value) {
            result.x = new_x;
            result.z = new_func_value;
        }
        if (maxRiter->x - r_i_value_previous_max->x <= eps_in_func) {
            st_flag = true;
        }
    }

    return result;
}

Point sequentialCalc(double left_x, double right_x,
                     double left_y, double right_y,
                     double(*func)(double x, double y)) {
    double eps = 0.01;
    double n_max_value = 1000;
    double(*ptr)(double, double) = func;
    double r = 2;

    Point result = {0, 0, 0};
    Point last_result = {0, 0, 0};
    std::set<doubleDimensionChar> set;

    result = singleDimensionMin(left_x, right_x, left_y, ptr);
    set.insert(doubleDimensionChar(result.x, result.y, result.z));
    last_result = result;

    result = singleDimensionMin(left_x, right_x, right_y, ptr);
    set.insert(doubleDimensionChar(result.x, result.y, result.z));

    if (result.z < last_result.z) { last_result = result;}

    int k = 2;
    bool stop = false;
    double mu, mu_current;
    double M;
    while (!stop && k < n_max_value) {
        // calc mu and M
        mu = -1;
        auto i_value = set.begin();
        i_value++;
        auto i_previous_value = set.begin();
        while (i_value != set.end()) {
            double delta_f = i_value->functionValue -
                                i_previous_value->functionValue;
            mu_current = std::abs(static_cast<double>((delta_f) /
            (i_value->y - i_previous_value->y)));
            if (mu_current > mu) {mu = mu_current;}
            i_value++; i_previous_value++;
        }
        if (mu > 0) {
            M = r * mu;
        } else {
            M = 1;
        }

        // calc R(i)
        // R(i) = m(x(i) - x(i-1)) + (z(i) - z(i-1))^2/ m(x(i)-x(i-1))
        // - 2(z(i)-z(i-1))
        i_value = set.begin();
        i_value++;
        i_previous_value = set.begin();
        double R, current_r;
        R = std::numeric_limits<double>::lowest();;

        auto maxRiter = set.begin();
        auto r_i_value_previous_max = set.begin();
        while (i_value != set.end()) {
            double delta_y = (i_value->y - i_previous_value->y);
            double delta_f = i_value->functionValue -
                                i_previous_value->functionValue;
            double a = M * delta_y;
            double b = pow((delta_f), 2) / a;
            double c = 2 * (delta_f);
            current_r = a + b - c;
            if (current_r > R) {
                R = current_r;
                maxRiter = i_value;
                r_i_value_previous_max = i_previous_value;
            }
            i_value++; i_previous_value++;
        }
        k++;

        double new_y = 0.5 * (maxRiter->y + r_i_value_previous_max->y) -
        ((maxRiter->functionValue - r_i_value_previous_max->functionValue) /
                                                                    (2 * M));

        result = singleDimensionMin(left_x, right_x, new_y, ptr);
        set.insert(doubleDimensionChar(result.x, result.y, result.z));
        if (result.z < last_result.z) {
            last_result = result;
        }
        if (maxRiter->y - r_i_value_previous_max->y <= eps) {
            stop = true;
        }
    }
    return last_result;
}

Point parralelCalc(double left_x, double right_x,
                   double left_y, double right_y,
                   double(*func)(double x, double y)) {
    int size, rank;
    MPI_Comm_size(MPI_COMM_WORLD, &size);
    MPI_Comm_rank(MPI_COMM_WORLD, &rank);

    if (size <= 1) {
        return sequentialCalc(left_x, right_x, left_y, right_y, func);
    }

    double eps = 0.01;
    double r = 2;
    int max_iterations = 1000;
    Point result = {0, 0, 0};
    Point last_result = {0, 0, 0};
    std::vector<double> result_sender(3);

    if (rank == 0) {
        int should_stop = 0;
        int stop_signal = 0;
        std::set<doubleDimensionChar> set;
        double len_of_part = (right_y - left_y) / (size-1);
        for (int i = 1; i < size; i++) {
            double y_value_send = left_y + i*len_of_part;
            MPI_Send(&stop_signal, 1, MPI_INT, i, 1, MPI_COMM_WORLD);
            MPI_Send(&y_value_send, 1, MPI_DOUBLE, i, 1, MPI_COMM_WORLD);
        }
        result = singleDimensionMin(left_x, right_x, left_y, func);
        set.insert(doubleDimensionChar(result.x, result.y, result.z));
        last_result = result;
        for (int i = 1; i < size ; i++) {
            MPI_Recv(result_sender.data(), 3, MPI_DOUBLE, MPI_ANY_SOURCE, 1,
                     MPI_COMM_WORLD, MPI_STATUS_IGNORE);
            result.x = result_sender[0]; result.y = result_sender[1];
            result.z = result_sender[2];
            if (result.z < last_result.z) { last_result = result; }
            set.insert(doubleDimensionChar(result.x, result.y, result.z));
        }

        int k = size;
        std::set<Interval> total_interval_set;
        double mu, mu_current;
        double M;
        while (should_stop == 0 && k < max_iterations) {
            auto set_iter = set.begin(); set_iter++;
            auto set_iter_previos = set.begin();

            mu = -1;
            while (set_iter != set.end()) {
                double delta_f = set_iter->functionValue -
                                set_iter_previos->functionValue;
                mu_current = std::abs(static_cast<double>((delta_f) /
                (set_iter->y - set_iter_previos->y)));
                if (mu_current > mu) {mu = mu_current;}
                set_iter++; set_iter_previos++;
            }
            if (mu > 0) {
                M = r * mu;
            } else {
                M = 1;
            }

            set_iter = set.begin(); set_iter++;
            set_iter_previos = set.begin();
            double current_r;

            total_interval_set.clear();
            while (set_iter != set.end()) {
                double delta_y = (set_iter->y - set_iter_previos->y);
                double delta_f = set_iter->functionValue -
                                set_iter_previos->functionValue;
                double a = M * delta_y;
                double b = pow((delta_f), 2) / a;
                double c = 2 * (delta_f);
                current_r = a + b - c;
                total_interval_set.insert(Interval(current_r,
                set_iter_previos->y, set_iter_previos->functionValue,
                set_iter->y, set_iter->functionValue));
                set_iter++; set_iter_previos++;
            }

            k += size;
            auto r_iter = total_interval_set.begin();
            for (int i = 1; i < size; i++) {
                double new_y = 0.5 *
                    (r_iter->variableFirst + r_iter->variableSecond) -
                    ((r_iter->functionValueSecond - r_iter->functionValueFirst)
                    / (2 * M));
                MPI_Send(&stop_signal, 1, MPI_INT, i, 1, MPI_COMM_WORLD);
                MPI_Send(&new_y, 1, MPI_DOUBLE, i, 1, MPI_COMM_WORLD);
                if (r_iter->variableSecond - r_iter->variableFirst <= eps) {
                    should_stop = 1;
                }
                r_iter++;
            }
            for (int i = 1; i < size; i++) {
                MPI_Recv(result_sender.data(), 3, MPI_DOUBLE, MPI_ANY_SOURCE, 1,
                         MPI_COMM_WORLD, MPI_STATUS_IGNORE);
                result.x = result_sender[0]; result.y = result_sender[1];
                result.z = result_sender[2];
                if (result.z < last_result.z) { last_result = result; }
                set.insert(doubleDimensionChar(result.x, result.y, result.z));
            }
        }
        stop_signal = 1;
        for (int i = 1; i < size; ++i) {
            MPI_Send(&stop_signal, 1, MPI_INT, i, 1, MPI_COMM_WORLD);
        }
    } else {
        while (true) {
            int stop_signal;
            MPI_Recv(&stop_signal, 1, MPI_INT, 0, 1,
                     MPI_COMM_WORLD, MPI_STATUS_IGNORE);
            if (stop_signal) {
                break;
            }
            double const_y;
            MPI_Recv(&const_y, 1, MPI_DOUBLE, 0, 1,
                     MPI_COMM_WORLD, MPI_STATUS_IGNORE);
            result = singleDimensionMin(left_x, right_x, const_y, func);
            result_sender[0] = result.x; result_sender[1] = result.y;
            result_sender[2] = result.z;
            MPI_Send(result_sender.data(), 3, MPI_DOUBLE,
                                           0, 1, MPI_COMM_WORLD);
        }
    }
    return last_result;
}

bool comparePoints(const Point& left, const Point& right) {
    const double eps = 0.01;
    bool dx = std::abs(left.x - right.x) <= eps;
    bool dy = std::abs(left.y - right.y) <= eps;
    bool dz = std::abs(left.z - right.z) <= eps;
    return dx && dy && dz;
}
\end{lstlisting}

main.cpp
\begin{lstlisting}
// Copyright 2021 Lakhov Kirill

#include <gtest/gtest.h>
#include <iostream>
#include "./optimization_params.h"
#include <gtest-mpi-listener.hpp>

TEST(Global_Params_Optimization_MPI, Test_Func1) {
    int rank;
    MPI_Comm_rank(MPI_COMM_WORLD, &rank);
    double left_x = -1;
    double right_x = 1;
    double left_y = -2;
    double right_y = 2;
    double(*func1)(double, double) = f1;
    double time1, time2, delta1, delta2;
    if (rank == 0) {
        time1 = MPI_Wtime();
    }
    Point parallel_result = parralelCalc(left_x, right_x,
                                                 left_y, right_y, func1);
    if (rank == 0) {
        time2 = MPI_Wtime();
        delta1 = time2 - time1;
        std::cout << "Parallel time " << delta1 << std::endl;
        time1 = MPI_Wtime();
        Point seq_result = sequentialCalc(left_x, right_x,
                                                  left_y, right_y, func1);
        time2 = MPI_Wtime();
        delta2 = time2 - time1;
        std::cout << "Sequental time " << delta2 << std::endl;
        std::cout << "sequental/parallel ratio " << delta2/delta1 << std::endl;
        bool r = comparePoints(parallel_result, seq_result);
        EXPECT_TRUE(r);
    }
}

TEST(Global_Params_Optimization_MPI, Test_Func2) {
    int rank;
    MPI_Comm_rank(MPI_COMM_WORLD, &rank);
    double left_x = -1;
    double right_x = 1;
    double left_y = -2;
    double right_y = 2;
    double(*func2)(double, double) = f2;
    double time1, time2, delta1, delta2;
    if (rank == 0) {
        time1 = MPI_Wtime();
    }
    Point parallel_result = parralelCalc(left_x, right_x,
                                                 left_y, right_y, func2);
    if (rank == 0) {
        time2 = MPI_Wtime();
        delta1 = time2 - time1;
        std::cout << "Parallel time " << delta1 << std::endl;
        time1 = MPI_Wtime();
        Point seq_result = sequentialCalc(left_x, right_x,
                                                  left_y, right_y, func2);
        time2 = MPI_Wtime();
        delta2 = time2 - time1;
        std::cout << "Sequental time " << delta2 <<std::endl;
        std::cout << "sequental/parallel ratio " << delta2/delta1 <<std::endl;
        bool r = comparePoints(parallel_result, seq_result);
        EXPECT_TRUE(r);
    }
}

TEST(Global_Params_Optimization_MPI, Test_Func3) {
    int rank;
    MPI_Comm_rank(MPI_COMM_WORLD, &rank);
    double left_x = 0.5;
    double right_x = 2;
    double left_y = 0.5;
    double right_y = 2;
    double(*func3)(double, double) = f3;
    double time1, time2, delta1, delta2;
    if (rank == 0) {
        time1 = MPI_Wtime();
    }
    Point parallel_result = parralelCalc(left_x, right_x,
                                                 left_y, right_y, func3);
    if (rank == 0) {
        time2 = MPI_Wtime();
        delta1 = time2 - time1;
        std::cout << "Parallel time " << delta1 << std::endl;
        time1 = MPI_Wtime();
        Point seq_result = sequentialCalc(left_x, right_x,
                                                  left_y, right_y, func3);
        time2 = MPI_Wtime();
        delta2 = time2 - time1;
        std::cout << "Sequental time " << delta2 << std::endl;
        std::cout << "sequental/parallel ratio " << delta2/delta1 << std::endl;
        bool r = comparePoints(parallel_result, seq_result);
        EXPECT_TRUE(r);
    }
}

TEST(Global_Params_Optimization_MPI, Test_Func4) {
    int rank;
    MPI_Comm_rank(MPI_COMM_WORLD, &rank);
    double left_x = 0;
    double right_x = 2;
    double left_y = 0;
    double right_y = 2;
    double time1, time2, delta1, delta2;
    double(*func4)(double, double) = f1;
    if (rank == 0) {
        time1 = MPI_Wtime();
    }
    Point parallel_result = parralelCalc(left_x, right_x,
                                                 left_y, right_y, func4);
    if (rank == 0) {
        time2 = MPI_Wtime();
        delta1 = time2 - time1;
        std::cout << "Parallel time "<< delta1 << std::endl;
        time1 = MPI_Wtime();
        Point seq_result = sequentialCalc(left_x, right_x,
                                                  left_y, right_y, func4);
        time2 = MPI_Wtime();
        delta2 = time2 - time1;
        std::cout << "Sequental time " << delta2 << std::endl;
        std::cout << "sequental/parallel ratio " << delta2/delta1 << std::endl;
        bool r = comparePoints(parallel_result, seq_result);
        EXPECT_TRUE(r);
    }
}

TEST(Global_Params_Optimization_MPI, Test_Func5) {
    int rank;
    MPI_Comm_rank(MPI_COMM_WORLD, &rank);
    double left_x = -1.5;
    double right_x = 1;
    double left_y = 0;
    double right_y = 1.5;
    double time1, time2, delta1, delta2;
    double(*func5)(double, double) = f1;
    if (rank == 0) {
        time1 = MPI_Wtime();
    }
    Point parallel_result = parralelCalc(left_x, right_x,
                                                 left_y, right_y, func5);
    if (rank == 0) {
        time2 = MPI_Wtime();
        delta1 = time2 - time1;
        std::cout << "Parallel time " << delta1 << std::endl;
        time1 = MPI_Wtime();
        Point seq_result = sequentialCalc(left_x, right_x,
                                                  left_y, right_y, func5);
        time2 = MPI_Wtime();
        delta2 = time2 - time1;
        std::cout << "Sequental time " << delta2 << std::endl;
        std::cout << "sequental/parallel ratio " << delta2/delta1 << std::endl;
        bool r = comparePoints(parallel_result, seq_result);
        EXPECT_TRUE(r);
    }
}

// Test for time measure
 TEST(Global_Params_Optimization_MPI, Test_Func6) {
     int rank;
     MPI_Comm_rank(MPI_COMM_WORLD, &rank);
     double left_x = -1;
     double right_x = 2;
     double left_y = -2;
     double right_y = 2;
     double(*func1)(double, double) = f1;
     double time1, time2, delta1, delta2;
     if (rank == 0) {
         time1 = MPI_Wtime();
     }
     Point parallel_result = parralelCalc(left_x, right_x,
                                                  left_y, right_y, func1);
     if (rank == 0) {
         time2 = MPI_Wtime();
         delta1 = time2 - time1;
         std::cout << "Parallel time " << delta1 << std::endl;
         time1 = MPI_Wtime();
         Point seq_result = sequentialCalc(left_x, right_x,
                                                   left_y, right_y, func1);
         time2 = MPI_Wtime();
         delta2 = time2 - time1;
         std::cout << "Sequental time " << delta2 << std::endl;
         std::cout << "sequental/parallel ratio "<< delta2/delta1
 << std::endl;
         bool r = comparePoints(parallel_result, seq_result);
         EXPECT_TRUE(r);
     }
 }

int main(int argc, char** argv) {
    ::testing::InitGoogleTest(&argc, argv);
    MPI_Init(&argc, &argv);
    ::testing::AddGlobalTestEnvironment(new GTestMPIListener::MPIEnvironment);
    ::testing::TestEventListeners& listeners =
        ::testing::UnitTest::GetInstance()->listeners();
    listeners.Release(listeners.default_result_printer());
    listeners.Release(listeners.default_xml_generator());
    listeners.Append(new GTestMPIListener::MPIMinimalistPrinter);
    return RUN_ALL_TESTS();
}
\end{lstlisting}

\end{document}