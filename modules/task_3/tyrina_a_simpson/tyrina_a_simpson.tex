\documentclass{report}

\usepackage[warn]{mathtext}
\usepackage[T2A]{fontenc}
\usepackage[utf8]{luainputenc}
\usepackage[english, russian]{babel}
\usepackage[pdftex]{hyperref}
\usepackage[12pt]{extsizes}
\usepackage{listings}
\usepackage{color}
\usepackage{geometry}
\usepackage{enumitem}
\usepackage{multirow}
\usepackage{graphicx}
\usepackage{indentfirst}
\usepackage{amsmath}

\geometry{a4paper,top=2cm,bottom=2cm,left=2.5cm,right=1.5cm}
\setlength{\parskip}{0.5cm}
\setlist{nolistsep, itemsep=0.3cm,parsep=0pt}

\usepackage{listings}
\lstset{language=C++,
        basicstyle=\footnotesize,
		keywordstyle=\color{blue}\ttfamily,
		stringstyle=\color{red}\ttfamily,
		commentstyle=\color{green}\ttfamily,
		morecomment=[l][\color{red}]{\#}, 
		tabsize=4,
		breaklines=true,
  		breakatwhitespace=true,
  		title=\lstname,       
}

\makeatletter
\renewcommand\@biblabel[1]{#1.\hfil}
\makeatother

\begin{document}

\begin{titlepage}

\begin{center}
Министерство науки и высшего образования Российской Федерации
\end{center}

\begin{center}
Федеральное государственное автономное образовательное учреждение высшего образования \\
Национальный исследовательский Нижегородский государственный университет им. Н.И. Лобачевского
\end{center}

\begin{center}
Институт информационных технологий, математики и механики
\end{center}

\vspace{4em}

\begin{center}
\textbf{\LargeОтчет по лабораторной работе} \\
\end{center}
\begin{center}
\textbf{\Large«Вычисление многомерных интегралов с использованием многошаговой схемы 
(метод Симпсона)»} \\
\end{center}

\vspace{4em}

\newbox{\lbox}
\savebox{\lbox}{\hbox{text}}
\newlength{\maxl}
\setlength{\maxl}{\wd\lbox}
\hfill\parbox{7cm}{
\hspace*{5cm}\hspace*{-5cm}\textbf{Выполнилa:} \\ студентка группы 381906-1 \\ Тырина А.А. \\
\\
\hspace*{5cm}\hspace*{-5cm}\textbf{Проверил:}\\ доцент кафедры МОСТ, \\ кандидат технических наук \\ Сысоев А. В.\\
}
\vspace{\fill}

\begin{center} Нижний Новгород \\ 2021 \end{center}

\end{titlepage}

\setcounter{page}{2}

% Содержание
\tableofcontents
\newpage

% Введение
\section*{Введение}
\addcontentsline{toc}{section}{Введение}
\par Численное интегрирование — вычисление значения определённого интеграла (как правило, приближённое). Под численным интегрированием понимают набор численных методов для нахождения значения определённого интеграла. Одним из таких методов является метод Симпсона (парабол).
\par Метод Симпсона заключается в интегрировании интерполяционного многочлена второй степени функции f(x) с узлами интерполяции a, b и m = (a+b)/2 — параболы p(x).Для повышения точности имеет смысл разбить отрезок интегрирования на N равных промежутков(по аналогии с методом трапеций), на каждом из которых применить метод Симпсона. 
\newpage

% Постановка задачи
\section*{Постановка задачи}
\addcontentsline{toc}{section}{Постановка задачи}
\par В данной лабораторной работе требуется реализовать последовательный алгоритм вычисления многомерных интегралов методом Симпсона и параллельный алгоритм с помощью библиотеки MPI.
\par Для оценки эффективности работы программы нужно произвести серию экспериментов, сравнивающих время выполнения последовательной и параллельной версии алгоритма, сравнить полученные результаты и сделать выводы.
\newpage

% Описание алгоритма
\section*{Описание алгоритма}
\addcontentsline{toc}{section}{Описание алгоритма}
\par
Метод Симпсона заключается в интегрировании интерполяционного многочлена второй степени функции f(x) с узлами интерполяции a, b и m = (a+b)/2 — параболы p(x).Для повышения точности имеет смысл разбить отрезок интегрирования на N равных промежутков(по аналогии с методом трапеций), на каждом из которых применить метод Симпсона.
Площадь параболы может быть найдена суммированием площадей 6 прямоугольников равной ширины. Высота первого из них должна быть равна f(a), с третьего по пятый — f(m), шестого — f(m). Таким образом, приближение методом Симпсона находим по формуле:
$\displaystyle {\int_{a}^{b} f(x) \, dx }\approx {\int_{a}^{b} P_{2}(x) \, dx}=\frac  {b-a}{6}{\left(f(a)+4f\left({\frac  {a+b}{2}}\right)+f(b)\right)}$.

\par Для вычисления многомерных интегралов воспользуемся многошаговой схемой.
\newpage

% Схема распараллеливания
\section*{Схема распараллеливания}
\addcontentsline{toc}{section}{Схема распараллеливания}
\par 
Распараллеливание происходит путем параллельного вычисления локальных сумм по формуле Симпсона. Каждый процесс вычисляет значения подынтегральной функции для своего числа отрезков, суммирует эти значения в свою локальную сумму. Затем функцией Reduce все локальные суммы складываются в глобальную в процессе с рангом 0. Останется только умножить результат на высоты оснований каждого интервала.
\newpage

% Описание программной реализации
\section*{Описание программной реализации}
\addcontentsline{toc}{section}{Описание программной реализации}
Программа состоит из заголовочного файла simpson.h и двух файлов исходного кода simpson.cpp и main.cpp.
\par В заголовочном файле находятся прототипы функций для последовательного и параллельного вычисления многомерных интегралов.
\par Функция для последовательного алгоритма:
\begin{lstlisting}
double getSequentialSimpson(function<double(vector<double>)> func,
                            vector<pair<double, double>> a_b, int n);
\end{lstlisting}
Входными параметрами функции является подынтегральная функция, массив пределов интегрирования, количество разбиений
\par Функция для параллельного
алгоритма:
\begin{lstlisting}
double getParallelSimpson(function<double(vector<double>)> f,
                          vector<pair<double, double>> limits, int n);
\end{lstlisting}
Входные параметры данной функции совпадают с входными параметрами функции для последовательного алгоритма.
\par В файле исходного кода simpson.cpp содержится реализация функций, объявленных в заголовочном файле. В файле исходного кода main.cpp содержатся тесты для проверки корректности программы.
\newpage

% Подтверждение корректности
\section*{Подтверждение корректности}
\addcontentsline{toc}{section}{Подтверждение корректности}
Для подтверждения корректности работы данной программы с помощью фрэймфорка Google Test была разработана серия тестов. В каждом из тестов вычисляется значение контрольного  интеграла заданной размерности при помощи последовательного и параллельного алгоритмов, подсчитывается время работы обоих алгоритмов, находится ускорение делением времени работы последовательного алгоритма на время работы параллельного. Результаты вычисления интеграла последовательным и параллельным способом сравниваются между собой в пределах погрешности, после чего можно сделать выводы об эффективнсти программы.

\par Успешное прохождение разработанных мной тестов подтверждает корректность работы программы.
\newpage

% Результаты экспериментов
\section*{Результаты экспериментов}
\addcontentsline{toc}{section}{Результаты экспериментов}
Вычислительные эксперименты для оценки эффективности работы параллельного алгоритма проводились на ПК со следующими характеристиками:
\begin{itemize}
\item Процессор: Intel Core i5-10210U, 3.5 ГГц, количество ядер: 8;
\item Оперативная память: 8 ГБ (DDR4), 3200  МГц;
\item Операционная система: Windows 10 Home.
\end{itemize}

\par Результаты экспериментов представлены в Таблице 1.

\begin{table}[!h]
\caption{Результаты вычислительных экспериментов в тесте 4}
\centering
\begin{tabular}{| p{2cm} | p{3cm} | p{4cm} | p{2cm} |}
\hline
Количество процессов & Время работы последовательного алгоритма (в секундах) & Время работы параллельного алгоритма (в секундах) & Ускорение  \\[5pt]
\hline
1        & 1.508        & 1.415     & 1.066       \\
2        & 1.403        & 0.822     & 1.707       \\
3        & 1.478        & 0.584     & 2.530       \\
4        & 1.494        & 0.459     & 3.254       \\
8        & 1.616        & 0.285     & 5.669	    \\
16        & 1.599        & 0.353     & 4.524	  \\
\hline
\end{tabular}
\end{table}
По данным, полученным в результате экспериментов, можно сделать вывод о том,
что параллельный алгоритм работает до 5 раз быстрее, чем последовательный.

\newpage

% Заключение
\section*{Заключение}
\addcontentsline{toc}{section}{Заключение}
В ходе выполнения данной лабораторной работы были разработаны последовательный и параллельный алгоритмы вычисления многомерных интегралов методом Симпсона. После выполенения серии экспериментов можно сделать вывод о том, что параллельный алгоритм работает быстрее последовательного, что доказывает эффективность параллельного алгоритма по сравнению с последовательным.
\par Были разработаны и доведены до успешного выполнения тесты,  с использованием GoogleC++ Testing Framework, которые  подтвердиди корректность выполнения  программы.
\newpage

% Литература
\section*{Литература}
\addcontentsline{toc}{section}{Литература}
\begin{enumerate}
\item Гергель В. П. Теория и практика параллельных вычислений. – 2007.
\itemГергель В. П., Стронгин Р. Г. Основы параллельных вычислений для многопроцессорных вычислительных систем. – 2003.

\end{enumerate} 
\newpage

% Приложение
\section*{Приложение}
\addcontentsline{toc}{section}{Приложение}

simpson.cpp
\begin{lstlisting}
// Copyright 2021 Tyrina Anastasia
#include "../../../modules/task_3/tyrina_a_simpson/simpson.h"

#include <mpi.h>

#include <cmath>

double getSequentialSimpson(function<double(vector<double>)> func,
                            vector<pair<double, double>> a_b, int n) {
  int integral_size = a_b.size();
  vector<double> h(integral_size);
  int num = 1;

  for (int i = 0; i < integral_size; ++i) {
    h[i] = (a_b[i].second - a_b[i].first) / n;
    num *= n;
  }

  double sum = 0.0;
  for (int i = 0; i < num; ++i) {
    vector<vector<double>> x(integral_size);
    int tmp = i;
    for (int j = 0; j < integral_size; ++j) {
      x[j].push_back(a_b[j].first + tmp % n * h[j]);
      for (int k = 0; k < 4; k++) {
        x[j].push_back(a_b[j].first + tmp % n * h[j] + h[j] / 2);
      }
      x[j].push_back(a_b[j].first + tmp % n * h[j] + h[j]);
      tmp /= n;
    }
    vector<double> comb;
    for (int i = 0; i < pow(6, integral_size); ++i) {
      int tmp = i;
      for (int j = 0; j < integral_size; ++j) {
        comb.push_back(x[j][tmp % 6]);
        tmp /= 6;
      }
      sum += func(comb);
      comb.clear();
    }
    x.clear();
  }
  for (int i = 0; i < integral_size; ++i) {
    sum *= h[i] / 6.0;
  }
  return sum;
}

double getParallelSimpson(function<double(vector<double>)> func,
                          vector<pair<double, double>> a_b, int n) {
  int size, rank;
  MPI_Comm_size(MPI_COMM_WORLD, &size);
  MPI_Comm_rank(MPI_COMM_WORLD, &rank);
  int integral_size = a_b.size();
  std::vector<double> h(integral_size);

  int global_num;
  if (rank == 0) {
    global_num = 1;
    for (int i = 0; i < integral_size; ++i) {
      h[i] = (a_b[i].second - a_b[i].first) / n;
      global_num *= n;
    }
  }
  MPI_Bcast(&global_num, 1, MPI_INT, 0, MPI_COMM_WORLD);
  MPI_Bcast(h.data(), integral_size, MPI_DOUBLE, 0, MPI_COMM_WORLD);

  int rem = global_num % size;
  int delta, start;
  if (rank == 0) {
    delta = global_num / size + rem;
    start = 0;
  } else {
    delta = global_num / size;
    start = rem + delta * rank;
  }

  double local_sum = 0.0;
  for (int i = start; i < delta + start; ++i) {
    vector<vector<double>> x(integral_size);
    int tmp = i;
    for (int j = 0; j < integral_size; ++j) {
      x[j].push_back(a_b[j].first + tmp % n * h[j]);
      for (int k = 0; k < 4; k++) {
        x[j].push_back(a_b[j].first + tmp % n * h[j] + h[j] / 2);
      }
      x[j].push_back(a_b[j].first + tmp % n * h[j] + h[j]);
      tmp /= n;
    }

    vector<double> comb;
    for (int i = 0; i < pow(6, integral_size); ++i) {
      int temp = i;
      for (int j = 0; j < integral_size; ++j) {
        comb.push_back(x[j][temp % 6]);
        temp /= 6;
      }
      local_sum += func(comb);
      comb.clear();
    }
    x.clear();
  }

  double global_sum = 0.0;
  MPI_Reduce(&local_sum, &global_sum, 1, MPI_DOUBLE, MPI_SUM, 0,
             MPI_COMM_WORLD);

  if (rank == 0) {
    for (int i = 0; i < integral_size; ++i) {
      global_sum *= h[i] / 6.0;
    }
  }
  return global_sum;
}

\end{lstlisting}
main.cpp
\begin{lstlisting}
// Copyright 2021 Tyrina Anastasia
#include <gtest/gtest.h>

#include <cmath>
#include <gtest-mpi-listener.hpp>

#include "./simpson.h"

const function<double(vector<double>)> func1 = [](vector<double> vec) {
  double x = vec[0];
  double y = vec[1];
  return x * x - 2 * y;
};

const double eps = 0.0001;

TEST(SIMPSON_METHOD_MPI, TEST_1) {
  int rank;
  MPI_Comm_rank(MPI_COMM_WORLD, &rank);

  std::vector<std::pair<double, double>> limits({{4, 10}, {1, 2}});
  int n = 100;

  double start = MPI_Wtime();
  double result = getParallelSimpson(func1, limits, n);
  double end = MPI_Wtime();

  if (rank == 0) {
    double ptime = end - start;

    start = MPI_Wtime();
    double reference_result = getSequentialSimpson(func1, limits, n);
    end = MPI_Wtime();
    double stime = end - start;

    std::cout << "Sequential: " << stime << std::endl;
    std::cout << "Parallel: " << ptime << std::endl;
    std::cout << "Speedup: " << stime / ptime << std::endl;
    ASSERT_NEAR(result, reference_result, eps);
  }
}

int main(int argc, char** argv) {
  ::testing::InitGoogleTest(&argc, argv);
  MPI_Init(&argc, &argv);

  ::testing::AddGlobalTestEnvironment(new GTestMPIListener::MPIEnvironment);
  ::testing::TestEventListeners& listeners =
      ::testing::UnitTest::GetInstance()->listeners();

  listeners.Release(listeners.default_result_printer());
  listeners.Release(listeners.default_xml_generator());

  listeners.Append(new GTestMPIListener::MPIMinimalistPrinter);
  return RUN_ALL_TESTS();
}

\end{lstlisting}

\end{document}